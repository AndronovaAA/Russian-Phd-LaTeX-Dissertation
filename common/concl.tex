%% Согласно ГОСТ Р 7.0.11-2011:
%% 5.3.3 В заключении диссертации излагают итоги выполненного исследования, рекомендации, перспективы дальнейшей разработки темы.
%% 9.2.3 В заключении автореферата диссертации излагают итоги данного исследования, рекомендации и перспективы дальнейшей разработки темы.
\begin{enumerate}
  \item Проведён анализ текущих методов робастного управления для шагающих роботов, вариантов формулирования математической модели для описания движения шагающего робота, способов задания неопределённостей в модели;
  \item Предложен метод математического моделирования неопределённой системы, описывающей динамику шагающего робота. Математическая модель, описывающая динамику шагающего робота, составляется таким образом, чтобы для неё можно было применить численные методы для нахождения оптимального робастного управления;
  \item Разработан численный метод для нахождения оптимального робастного управления для систем с мультипликативной неопределённостью и в матрицах состояния, управления и наблюдения, используя линейные матричные неравенства и ортогональную декомпозицию. Для данного метода используется наблюдатель Люенбергера для оценки вектора состояния;
  \item Разработан численный метод для нахождения оптимального робастного управления для систем с мультипликативной и аддитивной неопределённостью, используя линейные матричные неравенства и ортогональную декомпозицию. Для данного метода используется наблюдатель Люенбергера для оценки вектора состояния;
  \item Разработан численный метод для нахождения оптимального робастного управления для систем с аддитивной неопределённостью и динамической обратной связью, используя линейные матричные неравенства и ортогональную декомпозицию. Данный метод является обобщением методов с обратной связью;
  \item Реализован комплекс программ в составе: а) моделирование различных неопределённостей в системе; б) реализация разработанных численных методов для вычисления оптимального робастного управления для систем с различными неопределённостями;
  \item Выполнено экспериментальное исследование на основе комплекса программ с целью оценки эффективности сформированного методов нахождения оптимального робастного управления.
\end{enumerate}
