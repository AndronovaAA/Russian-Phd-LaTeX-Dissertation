%% Согласно ГОСТ Р 7.0.11-2011:
%% 5.3.3 В заключении диссертации излагают итоги выполненного исследования, рекомендации, перспективы дальнейшей разработки темы.
%% 9.2.3 В заключении автореферата диссертации излагают итоги данного исследования, рекомендации и перспективы дальнейшей разработки темы.
\begin{enumerate}
  \item В результате анализа научных публикаций, установлено, что имеющиеся методы основанные на линейных матричных неравенствах не обладают достаточной робастностью, так как не учитывают неопределённости одновременно во всех матрицах. Исходя из этого предложен новый метод построения линейных матричных неравенств для систем с мультипликативными неопределённостями и ортогональной декомпозиции.  
  \item Предложено прямой и обратной динамики шагающих роботов и описаны используемые в работе математические инструменты.
  \item Разработан метод нахождения оптимального робастного управления для систем с мультипликативной неопределённостью и в матрицах состояния, управления и наблюдения. 
  \item Разработан метод обобщающий предыдущий, где неопределённости не ограничены единичной матрицей, а находится наибольшие возможные неопределённости для стабилизации системы.
  \item Разработан метод нахождения оптимального робастного управления для систем с мультипликативной и аддитивной неопределённостью. Рассмотрены случаи для единичной и мягкой неопределённостей.
  \item Разработан метод нахождения оптимального робастного управления для систем с аддитивной неопределённостью и динамической обратной связью.
  \item Численные исследования показали, что разработанные методы выполняют обозначенную функцию нахождения оптимального робастного управления с помощью линейных матричных неравенств.
  \item Математическое моделирование показало, что предложенные методы могут быть применены для управления четвероногими шагающими роботами. 
\end{enumerate}
