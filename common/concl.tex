%% Согласно ГОСТ Р 7.0.11-2011:
%% 5.3.3 В заключении диссертации излагают итоги выполненного исследования, рекомендации, перспективы дальнейшей разработки темы.
%% 9.2.3 В заключении автореферата диссертации излагают итоги данного исследования, рекомендации и перспективы дальнейшей разработки темы.
\begin{enumerate}
  \item Проведён анализ существующей литературы 
  \item Предложен метод математического моделирования неопределённой системы, описывающей динамику шагающего робота;
  \item Разработан численный метод для нахождения оптимального робастного управления для систем с мультипликативной неопределённостью и в матрицах состояния, управления и наблюдения, используя линейные матричные неравенства;
  \item Разработан численный метод для нахождения оптимального робастного управления для систем с мультипликативной и аддитивной неопределённостью, используя линейные матричные неравенства;
  \item Разработан метод нахождения оптимального робастного управления для систем с мультипликативной и аддитивной неопределённостью. Рассмотрены случаи для единичной и мягкой неопределённостей.
  \item Разработан метод нахождения оптимального робастного управления для систем с аддитивной неопределённостью и динамической обратной связью.
  \item Численные исследования показали, что разработанные методы выполняют обозначенную функцию нахождения оптимального робастного управления с помощью линейных матричных неравенств.
  \item Реализован комплекс программ
\end{enumerate}
