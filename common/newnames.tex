% Новые переменные, которые могут использоваться во всём проекте
% ГОСТ 7.0.11-2011
% 9.2 Оформление текста автореферата диссертации
% 9.2.1 Общая характеристика работы включает в себя следующие основные структурные
% элементы:
% актуальность темы исследования;
\newcommand{\actualityTXT}{Актуальность темы исследования}
% степень ее разработанности;
\newcommand{\progressTXT}{Степень разработанности темы исследования}
% объект и предмет;
\newcommand{\objectsubjectTXT}{Объект и предмет исследования}
% цели и задачи;
\newcommand{\aimtasksTXT}{Цель и задачи диссертационной работы}
%\newcommand{\tasksTXT}{задачи}
% или чаще используют просто
%\newcommand{\influenceTXT}{Практическая значимость}
% методологию и методы исследования;
\newcommand{\methodsTXT}{Методы исследования}
% положения, выносимые на защиту;
\newcommand{\defpositionsTXT}{Положения, выносимые на~защиту}
% соответствие паспорту специальности.
\newcommand{\complianceTXT}{Соответствие результатов диссертации паспорту научной специальности}
% научную новизну;
\newcommand{\noveltyTXT}{Научная новизна результатов}
% степень достоверности результатов.
\newcommand{\probationTXT}{Достоверность и обоснованность результатов}
% теоретическую и практическую значимость работы;
\newcommand{\influenceTXT}{Теоретическая и практическая значимость}
% апробаця результатов
\newcommand{\reliabilityTXT}{Апробация результатов}
\newcommand{\publicationsTXT}{Публикации}
\newcommand{\contributionTXT}{Личный вклад автора}


%%% Заголовки библиографии:

% для автореферата:
\newcommand{\bibtitleauthor}{Публикации автора по теме диссертации}

% для стиля библиографии `\insertbiblioauthorgrouped`
\newcommand{\bibtitleauthorvak}{В изданиях из списка ВАК РФ}
\newcommand{\bibtitleauthorscopus}{В изданиях, входящих в международную базу цитирования Scopus}
\newcommand{\bibtitleauthorwos}{В изданиях, входящих в международную базу цитирования Web of Science}
\newcommand{\bibtitleauthorother}{В прочих изданиях}
\newcommand{\bibtitleauthorconf}{В сборниках трудов конференций}
\newcommand{\bibtitleauthorpatent}{Зарегистрированные патенты}
\newcommand{\bibtitleauthorprogram}{Зарегистрированные программы для ЭВМ}

% для стиля библиографии `\insertbiblioauthorimportant`:
\newcommand{\bibtitleauthorimportant}{Наиболее значимые \protect\MakeLowercase\bibtitleauthor}

% для списка литературы в диссертации и списка чужих работ в автореферате:
\newcommand{\bibtitlefull}{Список литературы} % (ГОСТ Р 7.0.11-2011, 4)
