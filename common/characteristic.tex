{\actuality} 

В последние десятилетия мобильным роботам уделяется большое внимание, поскольку они могут исследовать сложную окружающую среду, космос, участвовать в спасательных операциях, выполнять задачи без участия человека и т.д. Мобильные роботы можно разделить на три категории: колёсные, гусеничные и шагающие. Система передвижения робота --- важнейшая характеристика мобильной конструкции, которая зависит не только от рабочего пространства, но и от таких технических показателей, как манёвренность, управляемость, состояние местности, эффективность и устойчивость. Каждая система имеет свои преимущества и недостатки \cite{zhong2019}.

Преимущества шагающих роботов определяются возможностью выбора позы и походки, количеством ног и их структуры. К примеру, колёсные и гусеничные роботы могут эффективно работать на ровной местности, однако большинство из них не могут двигаться на захламлённой местности, в сложных и опасных условиях. Шагающие роботы обладают большими возможностями для передвижения практически по всей поверхности земли в различных условиях, они могут проходить там, где может человек и животное \cite{Silva2012}. В зависимости от условий окружающей среды животные могут адаптировать свою походку и передвигаться с высокой эффективностью, обладая отличной локомоцией и ловкостью. Поэтому разработка шагающих роботов вдохновлена живыми организмами.

Управление шагающими роботами находится в центре внимания исследователей уже несколько десятилетий. Шагающая локомоция объединяет дискретные события (моделирование контакта между роботом и окружающей средой) с непрерывной динамикой, что делает управление такими системами отличным от многих других известных задач управления. Дискретная природа шагающего движения часто явно закладывается в задаче планирования траектории \cite{katayama2022whole, lu2023whole}, а также в методах управления с прогнозирующими моделями, разработанных для шагающих роботов \cite{KIM2019, chignoli2021humanoid}. Даже когда задача управления формулируется для очень короткого временного горизонта (или одного момента времени), тот факт, что движение включает дискретные шаги, меняет и ограничивает предположения, которые мы можем сделать о модели робота.

Аналитические модели роботов часто бывают неточными и вызывают неопределённость в описании динамики. Сложный набор датчиков и несколько уровней программного обеспечения вносят шум и задержки в передачу информации. Обычные законы управления часто оказываются недостаточными для эффективного решения этих проблем. Специализированные методы управления, разработанные для решения этой сложной задачи, обычно требуют длительного процесса проектирования и длительной настройки параметров \cite{underactuated}. Методы на основе ортогонального разложения были разработаны для уменьшения размерности вычислений в работах \cite{Schur, mason2014full}.

Изменение контакта между роботом и окружающей средой приводит к изменению ограничений, накладываемых на динамику робота. В связи с этим, можно выбрать описание состояния робота в максимальных координатах, которые состоят из активных состояний, способных изменяться, и статических состояний, вынужденных оставаться постоянными из--за действия ограничений \cite{SAVIN2021}. Влияние статических состояний на модель аналогично постоянным внешним силам, действующим на систему. Однако они наследуют структуру от моделей робота и окружающей среды, что может быть использовано при проектировании управления. С другой стороны, они меняются при каждом изменении контакта, что делает многие из известных методов управления, которые работают для статических внешних сил, неприменимыми в данном случае.

Разработка робастного управления для шагающих роботов является важным направлением в современной робототехнике. Хотя было предложено множество эффективных способов разработки робастных регуляторов для линейных \cite{POLYAK2021, Nicolett2018} и нелинейных \cite{HAUSWIRTH2024, Celentano2018} систем, новые типы роботов ставят новые, часто более сложные проблемы робастного управления. Для мобильных роботов не всегда доступно хорошее описание окружающей среды. Кроме того, для шагающих роботов представление их динамики в виде обыкновенного дифференциального уравнения (ОДУ) хорошо работает только на коротких временных горизонтах, например, между последовательными шагами. В результате разработка робастного управления для этого типа роботов приобретает особую форму, которая ранее не рассматривалась в литературе. В данной работе мы решаем задачу робастного оптимального управления для математических моделей, описывающих динамику шагающих роботов с помощью ортогональных проекций.

Чтобы учесть неопределённость, формулируется модель неопределённости, состоящую из модели, которая гарантированно содержит неопределённость системы и удобна для работы, а затем проектируется система управления, которая удовлетворяет требованиям по устойчивости для всех других возможных моделей в модели неопределённости.

В реальных условиях измерение вектора состояния обычно нецелесообразно из--за необходимости устанавливать датчики в труднодоступных местах, измерять производные высокого порядка и т.д. Измерение возмущений является ещё более сложной задачей. Преодолеть (или смягчить) эти трудности можно, если полностью использовать имеющуюся предварительную информацию о модели и текущих измерениях её входов и выходов. Оценка состояния необходима, когда значения некоторых переменных состояния, требуемых законом управления не доступны непосредственно из измерений. Это может быть результатом отсутствия датчиков, шума или выбранной параметризации системы. В типичном сценарии закон управления требует полной информации о состоянии, в то время как только некоторые переменные состояния непосредственно измеряются \cite{Ackermann2001}. В данной диссертационной работе предложены численные методы включающие наблюдатель Люенбергера, а также рассмотрен обобщающий метод для систем с динамической обратной связью.

В данной работе предложены методы проектирования робастного управления и наблюдателей, которые автоматически доказывают устойчивость линейной модели. В то время как существует ряд работ по проектированию управления и наблюдателей для шагающих роботов с точной информацией о модели, вопросы проектирования робастного управления (при мультипликативной неопределённости в модели) в литературе не рассматривались. Также в отличие от существующих методов робастного управления общего назначения, мы одновременно рассматриваем структурированные мультипликативные и аддитивные неопределённости модели.

{\progress}

История линейных матричных неравенств в рамках анализа динамических систем начинается с теории Ляпунова. Он показал, что дифференциальное уравнение
\begin{align*}
	\frac{\partial}{\partial t}x(t)=Ax(t),
\end{align*}
устойчиво тогда и только тогда, когда существует положительно определённая матрица $P > 0$ такая, что
\begin{align*}
	A^T P + P A < 0.
\end{align*}

Следующее важное развитие теории было сделано в 1940-х годах Лурье, Постниковым и другими \cite{LMI1}. 
Они применили методы Ляпунова к практическим задачам управления и вывели линейные матричные неравенства, которые решались аналитически вручную, что ограничивало их применение. 

Следующим важным открытием стало то, что мы сейчас называем леммой о положительно--вещественных матрицах. Она была выведена Калманом, Якубовичем и Поповым в начале 1960-х годов. Им удалось свести решение линейных матричных неравенств, возникающих в задаче Лурье, к простым графическим критериям. Их вклад в \cite{LMI2} заключался в описании способа решения определённого семейства линейных матричных неравенств графическими методами.

Пятницкий и Скородинский были теми, кто сформулировал поиск функции Ляпунова как выпуклую оптимизационную задачу, а затем применили алгоритм, гарантированно решающий её в \cite{LMI3}.
В последние десятилетия использование линейных матричных неравенств для проектирования различных регуляторов становится всё более популярным. Различные методы \cite{LMI4, LMI5} были объединены с линейными матричными неравенствами, чтобы упростить проектирование регуляторов и продемонстрировать устойчивость управляемой системы.

В последние годы были сделаны различные предложения по достижению робастной устойчивости \cite{LMI7, LMI8}, но их разработка может потребовать больших вычислительных затрат.
Устойчивость по конечному времени также может быть достигнута с помощью линейных матричных неравенств, результаты, представленные в работе \cite{LMI6}, используются для работы с поведением динамических линейных систем «вход--выход». В книге \cite{Amato2011} представлены задачи конечной устойчивости линейных систем, решаемые с помощью линейных матричных неравенств.

Одним из способов оценки возмущений, поступающих извне системы, является применение наблюдателя возмущений. Его основная задача --- оценить возмущение на входном этапе, которое затем может быть использовано для борьбы с внешним возмущением. Были разработаны различные теоретические методы построения наблюдателей, большинство из которых рассмотрены в обзорах \cite{ObserverITMO} и \cite{Disturb_obs}. Базовая конструкция регулятора не имеет ограничений, пока она стабилизирует систему при включении наблюдателя возмущений во внутренний контур обратной связи. В линейно--квадратичном регуляторе оптимальный коэффициент усиления регулятора с обратной связью по состоянию может быть получен с помощью алгебраических уравнений Риккати, которые впоследствии используются при проектировании наблюдателя. В работах \cite{LQR1} и \cite{LQR2} был разработан регулятор на основе наблюдателя с использованием линейно--квадратичного регулятора для систем с различными возмущениями. Этот подход широко применим к классу задач оптимального управления со структурными ограничениями. 

В данной работе предлагается использовать наблюдатель для работы с явными механическими ограничениями в виде низкоразмерного наблюдателя состояния, основанного на ортогональных проекциях. Вдохновением для этого послужили работы \cite{SAVIN2021, Righetti2011}, где ортогональные проекции были использованы для замены дифференциальных алгебраических уравнений (ДАУ) для описания динамики эквивалентным ОДУ. 

{\objectsubject} 

Объектом исследования является оптимальное робастное управление шагающими роботами с мультипликативными и аддитивными неопределённостями.

Предметом исследования являются методы математического моделирования системы с неопределённостями, описывающие динамику шагающего робота и численные методы с использованием ортогональных методов и линейных матричных неравенств для нахождения коэффициентов регулятора и наблюдателя с последующими вычислениями при помощи комплекса программ.

{\aimtasks} 

Целью диссертационной работы является разработка численных ортогональных методов оптимального робастного управления математическими моделями, описывающих шагающих роботов, с использованием линейных матричных неравенств.

Для достижения поставленной цели необходимо было решить следующие задачи:
\begin{enumerate}[beginpenalty=10000] % https://tex.stackexchange.com/a/476052/104425
	\item Анализ текущих методов оптимального робастного управления для шагающих роботов, вариантов формулирования математической модели для описания движения шагающего робота, способов задания неопределённостей в модели;
	\item Разработка и исследование метода моделирования неопределённой динамической системы шагающего робота; 
	\item Разработка метода оптимального робастного управления для динамической системы с мультипликативными неопределённостями и статичной обратной связью;
	\item Разработка метода оптимального робастного управления для системы с мультипликативными и аддитивными неопределённостями и статичной обратной связью;
	\item Разработка численного метода оптимального робастного управления для системы, описывающую шагающего робота, с аддитивными неопределённостями и управлением с обратной динамической связью;
	\item Численный анализ разработанных линейных матричных неравенств для нахождения коэффициентов регулятора и наблюдателя для робастного управления, решение прямой задачи динамики, исследование зависимостей результатов от параметров оптимизации;
	\item Реализация комплекса программных средств в составе: а) моделирование мультипликативных неопределённостей в системе; б) реализация разработанных численных методов для вычисления оптимального робастного управления. Выполнение численного исследования на основе комплекса программ с целью оценки эффективности сформированных методов нахождения оптимального робастного управления.
\end{enumerate}

{\methods} 

Для решения вышеуказанных задач и достижения цели использовались методы теории управления и устойчивости, дифференциальных уравнений, численно--аналитические, выпуклой оптимизации, методы математического моделирования. Программы написаны на языке программирования Python.

{\defpositions}

\begin{enumerate}[beginpenalty=10000] % https://tex.stackexchange.com/a/476052/104425
	\item Численный метод оптимального робастного управления для системы с мультипликативными неопределённостями и обратной связью по полному состоянию. Позволяет находить оптимальные коэффициенты регулятора и наблюдателя для различных неопределённостей в системе, удовлетворяющих заданным условиям и с использованием линейных матричных неравенств; 
	\item Численный метод оптимального робастного управления для системы с аддитивными неопределённостями и с динамической обратной связью. Позволяет находить оптимальные коэффициенты регулятора для различных неопределённостей, удовлетворяющих заданным условиям и с использованием линейных матричных неравенств; 
	\item Комплекс программных средств в составе: а) моделирование мультипликативных неопределённостей в системе; б) реализация разработанных численных методов для вычисления оптимального робастного управления. 
\end{enumerate}

{\compliances} 

Содержание диссертации соответствует пунктам паспорта научной специальности 1.2.2 --- «Математическое моделирование, численные методы и комплексы программ» (физико--математические науки):
\begin{enumerate}[beginpenalty=10000]
	\item Пункт 2. Разработка, обоснование и тестирование эффективных вычислительных методов с применением современных компьютерных технологий;
	\item Пункт 3. Реализация эффективных численных методов и алгоритмов в виде комплексов проблемно--ориентированных программ для проведения вычислительного эксперимента;
	\item Пункт 8. Комплексные исследования научных и технических проблем с применением современной технологии математического моделирования и вычислительного эксперимента.
\end{enumerate}

{\novelty}

Основные результаты, полученные в работе, являются новыми. В частности
\begin{enumerate}[beginpenalty=10000] % https://tex.stackexchange.com/a/476052/104425
	\item Впервые разработаны численные методы робастного оптимального управления с обратной связью по полному состоянию, используя линейные матричные неравенства для систем с механическими ограничениями и мультипликативными неопределённостями;
	\item Впервые разработаны численные методы робастного оптимального управления с динамической обратной связью, используя линейные матричные неравенства для систем с механическими ограничениями и аддитивными неопределённостями;
	\item Предложен метод математического моделирования для неопределённых систем, описывающих динамику шагающих роботов, включающий ортогональные методы и линеаризацию;
	\item Реализован комплекс программ в составе: а) моделирование мультипликативных неопределённостей в системе; б) реализация разработанных численных методов для вычисления оптимального робастного управления. Выполнено экспериментальное исследование на основе комплекса программных средств с целью анализа и оценки сформированных методов.
\end{enumerate}

{\probation}

Достоверность полученных результатов обеспечивается строгими математическими доказательствами теоретических результатов, корректностью исходных и упрощающих допущений для численного вычисления, адекватными результатами вычислительных экспериментов, использованием уравнений, методов и подходов, строго обоснованных в научной литературе, апробированных и хорошо зарекомендованных при разработке робастного управления для систем с неопределённостями. Разработка программного комплекса для реализации практического использования предложенных методов производилась на основе современных подходов проектирования программных систем.

{\influence} 

Теоретическая значимость данной работы состоит в разработке нового метода для вычисления коэффициентов регулятора и наблюдателя для робастного управления системы с мультипликативными неопределённостями с использованием линейных матричных неравенств. Разработанный численный метод для вычисления коэффициентов регулятора и наблюдателя вносит вклад в развитие теории выпуклой оптимизации и теории управления.

Разработанные математические модели, численные методы и комплекс программ имеют практическую значимость для эксплуатации шагающих роботов. Практическая значимость результатов подтверждается применением результатов теоретической части в практике: приводятся формулировки задач оптимизационного управления для поиска коэффициентов регуляторов и наблюдателей для управления динамических систем, описывающих динамику шагающего робота. 

{\reliability} 

Результаты, включённые в данную работу, опубликованы в 1 печатном изданиях, 1 из которых изданы в журналах, рекомендованных ВАК.
\begin{comment}
, 1 --- в периодических научных журналах, индексируемых \textit{Web of Science} и \textit{Scopus}. 
\end{comment}

Результаты находятся в соответствии с результатами, полученными другими авторами. Результаты работы неоднократно докладывались на научно--методических семинарах и были использованы в учебной программе Университета Иннополис.

{\contribution} 

Все результаты, выносимые на защиту, получены лично автором: выбор методик решения задач, разработка алгоритма построения оптимального робастного управления, написание программ для моделирования шагающих роботов и использования разработанных численных экспериментов и анализ результатов расчётов. Постановка задач, обсуждение и интерпретация полученных результатов осуществлялась совместно с научным руководителем.