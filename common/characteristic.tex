
{\actuality} 
Разработка шагающих роботов одно из сильно развивающихся направлений в робототехнике. Шагающие роботы имеют множество применений в реальной жизни. От передвижения по сложной поверхности до участия в съёмках фильма.

Разработка робастного управления для шагающих роботов является важным направлением в современной робототехнике. Исследования в области робастного управления были мотивированы, в частности, применением в робототехнике. Хотя было предложено множество эффективных способов разработки робастных контроллеров для линейных \cite{POLYAK2021,Nicolett2018} и нелинейных \cite{HAUSWIRTH2024,Celentano2018} систем, новые типы роботов ставят новые, часто более сложные версии проблемы робастного управления. Для мобильных роботов не всегда доступно хорошее описание окружающей среды. Кроме того, для шагающих роботов представление их динамики в виде обыкновенного дифференциального уравнения (ОДУ) хорошо работает только на коротких временных горизонтах, например, между последовательными шагами. В результате разработка робастного управления для этого типа роботов приобретает особую форму, которая ранее не рассматривалась в литературе. В данной работе мы решаем задачу робастного управления с комбинацией аддитивных и мультипликативных неопределённостей, связанную с шагающими роботами.

Управление шагающими роботами находится в центре внимания исследователей уже несколько десятилетий. Шагающая локомоция объединяет дискретные события (приобретение и потеря контакта между роботом и окружающей средой) с непрерывной динамикой, что делает управление такими системами отличным от многих других известных задач управления. Дискретная природа шагающего движения часто явно кодируется в задаче планирования траектории \cite{katayama2022whole, lu2023whole}, а также в методах управления с прогнозирующими моделями, разработанных для шагающих роботов \cite{KIM2019, chignoli2021humanoid}. Даже когда задача управления формулируется для очень короткого временного горизонта (или одного момента времени), тот факт, что движение включает дискретные шаги, меняет и ограничивает предположения, которые мы можем сделать о модели робота.

Изменение контакта между роботом и окружающей средой приводит к изменению ограничений, накладываемых на динамику робота. В связи с этим мы можем выбрать описание состояния робота в максимальных координатах, которые состоят из активных состояний, способных изменяться, и статических состояний, вынужденных оставаться постоянными из-за действия ограничений \cite{SAVIN2021}. Влияние статических состояний на модель сродни постоянным внешним силам. Однако они наследуют структуру от моделей робота и окружающей среды, что может быть использовано при проектировании управления на основе моделей. С другой стороны, они меняются при каждом изменении контакта, что делает многие из известных методов управления, которые работают для статических внешних сил, неприменимыми в данном случае.

В данной работе мы предлагаем методы проектирования робастного управления и наблюдателей, которые автоматически доказывают устойчивость линеаризованной модели. В то время как существует ряд работ по проектированию управления и наблюдателей для шагающих роботов с точной информацией о модели, вопросы проектирования робастного управления (в смысле мультипликативной неопределённости модели) в литературе не рассматривались. В отличие от существующих методов робастного управления общего назначения, мы одновременно рассматриваем структурированные мультипликативные и аддитивные неопределённости модели.
Методы на основе ортогонального разложения были разработаны для уменьшения размерности вычислений Шуром и другими.

Методы на основе линейных матричных неравенств для поиска коэффициентов регулятора и наблюдателей были впервые использованы ... и затем ....

% {\progress}
% Этот раздел должен быть отдельным структурным элементом по
% ГОСТ, но он, как правило, включается в описание актуальности
% темы. Нужен он отдельным структурынм элемементом или нет ---
% смотрите другие диссертации вашего совета, скорее всего не нужен.

{\progress}
Этот раздел должен быть отдельным структурным элементомdfg по

{\objectsubject} \ldots

{\aimtasks} 
Целью диссертационной работы является разработка ортогональных методов оптимального робастного управления шагающими роботами с использованием матричных неравенств.

Для~достижения поставленной цели необходимо было решить следующие задачи:
\begin{enumerate}[beginpenalty=10000] % https://tex.stackexchange.com/a/476052/104425
	\item Исследовать текущие методы робастного управления для шагающих роботов.
	\item Разработать методы
	\item Подтвердить верность разработанных методов на математической модели шагающего робота.
\end{enumerate}

{\methods} \ldots

{\defpositions}
\begin{enumerate}[beginpenalty=10000] % https://tex.stackexchange.com/a/476052/104425
	\item Первое положение
	\item Второе положение
	\item Третье положение
	\item Четвёртое положение
\end{enumerate}

{\compliances} \ldots
Содержание диссертации соответствует пунктам (напишите номера и формулировку
пунктов) паспорта научной специальности 1.2.2 — «Математическое моделирование,
численные методы и комплексы программ».

{\novelty}
Основные результаты, полученные в работе, являются новыми. В частности
\begin{enumerate}[beginpenalty=10000] % https://tex.stackexchange.com/a/476052/104425
	\item Впервые \ldots
	\item Впервые \ldots
	\item Было выполнено оригинальное исследование \ldots
\end{enumerate}

{\probation}
Основные результаты работы докладывались~на:
перечисление основных конференций, симпозиумов и~т.\:п.

{\influence} \ldots

{\reliability} полученных результатов обеспечивается \ldots \ Результаты находятся в соответствии с результатами, полученными другими авторами.

{\contribution} Все результаты, выносимые на защиту, получены лично автором: выбор методик решения задач, разработка алгоритма ... , написание программ для моделирования шагающих роботов и использования разработанных численных экспериментов, графическую разработку и анализ результатов расчётов. Постановка задач, обсуждение и интерпретация полученных результатов осуществлялась совместно с научным руководителем.