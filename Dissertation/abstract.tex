\chapter*{Аннотация}                         % Заголовок
\addcontentsline{toc}{chapter}{Аннотация} 

В диссертационной работе рассматривается задача управления системами с явными ограничениями (такими как механические системы с контактным взаимодействием и их линеаризованные модели) при наличии параметрической неопределённости, в том числе в самих ограничениях. Разработан метод поиска оптимального закона управления (включая синтез оптимального наблюдателя состояния), обеспечивающего устойчивость для такого рода систем при наличии аддитивной неопределённости и мультипликативной неопределённости ограниченной по норме. Найдены достаточные условия для устойчивости замкнутой системы, записываемые в виде линейных матричных неравенств с параметром. Результирующий метод позволяет сформулировать задачу поиска оптимального управления как серию задач выпуклой оптимизации и использовать численные методы для их решения гарантируя высокую точность и быструю сходимость к оптимальному решению. Был предложен более общий метод, позволяющий автоматически находить релаксацию описанных выше задач в случае, если параметрическая неопределённость слишком велика для существования закона управления обеспечивающего робастную устойчивость. Предложенный метод автоматически находит наибольшее подмножество моделей для которых может быть гарантирована устойчивость (путём уменьшения нормы, ограничивающей мультипликативную неопределённость). Данный метод является инструментом анализа моделей систем с неопределённостью, указывая на требования к точности математической модели с позиций возможности синтеза робастного закона управления. Был предложен метод поиска закона управления, робастного к максимальному множеству моделей (или, эквивалентно, к модели с мультипликативной неопределённостью максимальной по норме). Был разработан метод автоматического синтеза регулятора с динамической обратной связью, снимая ограничения на структуру наблюдателя состояния.  Валидность всех предложенных методов была доказана в форме теорем и проведена численными экспериментами на моделях механических систем с явными ограничениями, в частности на модели четырёхногого шагающего робота.