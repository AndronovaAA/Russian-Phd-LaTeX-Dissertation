\chapter*{Аннотация}                         % Заголовок
\addcontentsline{toc}{chapter}{Аннотация} 

Диссертация посвящена формулированию численного метода оптимального робастного управления для систем с мультипликативными неопределённостями, ограниченными по норме, и обратной связью. Метод позволяет найти оптимальные коэффициенты регулятора и наблюдателя гарантирующие устойчивость. В работе также приводится обобщающий метод, где радиус неопределённости является переменным, что позволяет, например, релаксировать проблему оптимизационного робастного управления и найти подходящий радиус неопределённости для задач, которые ранее не имели решения. Данная формулировка оптимизационной проблемы позволяет разработать регулятор, который будет робастным по максимальному количеству задач управления. В работе также приведены методы робастного оптимального управления для систем со смешанными (мультипликативными и аддитивными) неопределённостями и динамической обратной связью. Для формулирования методов были использованы линейные матричные неравенства для поиска оптимального решения с помощью инструментов выпуклой оптимизации. Были проведены численные эксперименты, подтвердившие достоверность предложенных методов.