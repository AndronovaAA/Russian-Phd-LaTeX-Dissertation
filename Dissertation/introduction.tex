\chapter*{Введение}                         % Заголовок
\addcontentsline{toc}{chapter}{Введение}    % Добавляем его в оглавление

\newcommand{\actuality}{\textbf{\actualityTXT}}
\newcommand{\progress}{\textbf{\progressTXT}}
\newcommand{\objectsubject}{\textbf{\objectsubjectTXT}}
\newcommand{\aimtasks}{\textbf{\aimtasksTXT}}
\newcommand{\methods}{\textbf{\methodsTXT}}
\newcommand{\defpositions}{\textbf{\defpositionsTXT}}
\newcommand{\compliances}{\textbf{\complianceTXT}}
\newcommand{\novelty}{\textbf{\noveltyTXT}}
\newcommand{\probation}{\textbf{\probationTXT}}
\newcommand{\influence}{\textbf{\influenceTXT}}
\newcommand{\reliability}{\textbf{\reliabilityTXT}}
\newcommand{\publications}{\textbf{\publicationsTXT}}
\newcommand{\contribution}{\textbf{\contributionTXT}}

{\actuality} 

В последние десятилетия мобильным роботам уделяется большое внимание, поскольку они могут исследовать сложную окружающую среду, космос, участвовать в спасательных операциях, выполнять задачи без участия человека и т.д. Мобильные роботы можно разделить на три категории; колёсные, гусеничные и шагающие. Система передвижения робота --- важнейшая характеристика мобильной конструкции, которая зависит не только от рабочего пространства, но и от таких технических показателей, как манёвренность, управляемость, состояние местности, эффективность и устойчивость. Каждая система имеет свои преимущества и недостатки \cite{zhong2019}.

Преимущества передвижения на ногах зависят от позы, количества ног и их функциональности. К примеру, колёсные и гусеничные роботы могут работать на ровной местности, однако большинство из них не могут работать на захламлённой местности, в сложных и опасных условиях. Шагающий робот обладает большим потенциалом для передвижения практически по всей поверхности земли в различных условиях, как и человек, и животное \cite{Silva2012}. Животные используют свои ноги для быстрого и надёжного передвижения по различным местностям, обладая отличной локомоцией и ловкостью. Чаще всего они передвигаются с высокой скоростью и эффективностью в зависимости от условий окружающей среды.

Управление шагающими роботами находится в центре внимания исследователей уже несколько десятилетий. Шагающая локомоция объединяет дискретные события (приобретение и потеря контакта между роботом и окружающей средой) с непрерывной динамикой, что делает управление такими системами отличным от многих других известных задач управления. Дискретная природа шагающего движения часто явно закладывается в задаче планирования траектории \cite{katayama2022whole, lu2023whole}, а также в методах управления с прогнозирующими моделями, разработанных для шагающих роботов \cite{KIM2019, chignoli2021humanoid}. Даже когда задача управления формулируется для очень короткого временного горизонта (или одного момента времени), тот факт, что движение включает дискретные шаги, меняет и ограничивает предположения, которые мы можем сделать о модели робота.

Аналитические модели роботов часто бывают неточными и вызывают неопределённость в динамике. Сложный набор датчиков и несколько уровней программного обеспечения вносят шум и задержки в передачу информации. Обычные законы управления часто оказываются недостаточными для эффективного решения этих проблем. Специализированные методы управления, разработанные для решения этой сложной задачи, обычно требуют длительного процесса проектирования и сложной настройки параметров. 

Разработка робастного управления для шагающих роботов является важным направлением в современной робототехнике. Хотя было предложено множество эффективных способов разработки робастных регуляторов для линейных \cite{POLYAK2021, Nicolett2018} и нелинейных \cite{HAUSWIRTH2024, Celentano2018} систем, новые типы роботов ставят новые, часто более сложные версии проблемы робастного управления. Для мобильных роботов не всегда доступно хорошее описание окружающей среды. Кроме того, для шагающих роботов представление их динамики в виде обыкновенного дифференциального уравнения (ОДУ) хорошо работает только на коротких временных горизонтах, например, между последовательными шагами. В результате разработка робастного управления для этого типа роботов приобретает особую форму, которая ранее не рассматривалась в литературе. В данной работе мы решаем задачу робастного управления с комбинацией аддитивных и мультипликативных неопределённостей, связанную с шагающими роботами.

Изменение контакта между роботом и окружающей средой приводит к изменению ограничений, накладываемых на динамику робота. В связи с этим мы можем выбрать описание состояния робота в максимальных координатах, которые состоят из активных состояний, способных изменяться, и статических состояний, вынужденных оставаться постоянными из-за действия ограничений \cite{SAVIN2021}. Влияние статических состояний на модель сродни постоянным внешним силам, действующим на систему. Однако они наследуют структуру от моделей робота и окружающей среды, что может быть использовано при проектировании управления. С другой стороны, они меняются при каждом изменении контакта, что делает многие из известных методов управления, которые работают для статических внешних сил, неприменимыми в данном случае.

Оценка состояния необходима, когда значения некоторых переменных состояния, требуемых законом управления не доступны непосредственно из измерений. Это может быть результатом отсутствия датчиков, шума измерений или выбранной параметризации системы. В типичном сценарии закон управления требует полной информации о состоянии, в то время как только некоторые переменные состояния непосредственно измеряются \cite{Ackermann2001}.

Чтобы учесть такую неопределённость, мы сначала придумаем модель неопределённости, состоящую из модели, которая гарантированно содержит неопределённость системы, и с ней легче работать, а затем спроектировать систему управления, которая удовлетворяет требованиям по устойчивости для всех других возможных моделей в модели неопределённости.

В данной работе мы предлагаем методы проектирования робастного управления и наблюдателей, которые автоматически доказывают устойчивость линеаризованной модели. В то время как существует ряд работ по проектированию управления и наблюдателей для шагающих роботов с точной информацией о модели, вопросы проектирования робастного управления (при мультипликативной неопределённости в модели) в литературе не рассматривались. Также в отличие от существующих методов робастного управления общего назначения, мы одновременно рассматриваем структурированные мультипликативные и аддитивные неопределённости модели.
Методы на основе ортогонального разложения были разработаны для уменьшения размерности вычислений Шуром и другими.

{\progress}

История линейных матричных неравенств в рамках анализа динамических систем начинается с теории Ляпунова. Он показал, что дифференциальное уравнение
\begin{align*}
	\frac{\partial}{\partial t}x(t)=Ax(t)
\end{align*}
устойчиво тогда и только тогда, когда существует положительно определённая матрица $P$ такая, что
\begin{align*}
	A^T P + P A <0.
\end{align*}

Следующее важное развитие теории было сделано в 1940-х годах Лурье \cite{LMI1}, Постниковым и другими. 
Они применили методы Ляпунова к практическим задачам техники управления и вывели линейные матричные неравенства, которые решались аналитически вручную, что ограничивало их применение. 

Следующим важным открытием стало то, что мы сейчас называем леммой о положительно-вещественных матрицах. Она была выведена Калманом, Якубовичем и Поповым в начале 1960-х годов. Им удалось свести решение линейных матричных неравенств, возникающих в задаче Лурье, к простым графическим критериям. Их вклад в \cite{LMI2} заключался в указании способа решения определённого семейства линейных матричных неравенств графическими методами.

Пятницкий и Скородинский \cite{LMI3} были теми, кто сформулировал поиск функции Ляпунова как выпуклую оптимизационную задачу, а затем применили алгоритм, гарантированно решающий её.
В последние десятилетия использование линейных матричных неравенств для проектирования различных регуляторов становится все более популярным. Различные методы \cite{LMI4, LMI5} были объединены с линейными матричными неравенствами, чтобы упростить проектирование регуляторов и продемонстрировать устойчивость управляемой системы.

В последние годы были сделаны различные предложения по достижению робастной устойчивости \cite{LMI7, LMI8}, но их разработка может потребовать больших вычислительных затрат.
Устойчивость по конечному времени также может быть достигнута с помощью линейных матричных неравенств, результаты, представленные в работе \cite{LMI6}, используются для работы с поведением динамических линейных систем «вход-выход». В книге \cite{Amato2011} в едином ключе представлены задачи конечной устойчивости линейных систем, решаемые с помощью линейных матричных неравенств.

Другим способом оценки возмущений, поступающих извне системы, является применение наблюдателя возмущений. Его основная задача --- оценить возмущение на входном этапе, которое затем может быть использовано для борьбы с внешним возмущением. Базовая конструкция регулятора не имеет ограничений, пока она стабилизирует систему при включении наблюдателя возмущений во внутренний контур обратной связи.

Были разработаны различные теоретические методы построения наблюдателей возмущений, большинство из которых рассмотрены в обзорах \cite{ObserverITMO} и \cite{Disturb_obs}. Но в данной работе предлагается использовать наблюдатель для работы с явными механическими ограничениями в виде низкоразмерного наблюдателя состояния, основанного на ортогональных проекциях. Вдохновением для этого послужили работы \cite{SAVIN2021, Righetti2011}, где ортогональные проекции были использованы для замены дифференциальных алгебраических уравнений (ДАУ) для описания динамики эквивалентным ОДУ. 

В линейно--квадратичном регуляторе оптимальный коэффициент усиления регулятора с обратной связью по состоянию может быть получен с помощью алгебраических уравнений Риккати, которые впоследствии используются при проектировании наблюдателя. В работах \cite{LQR1} и \cite{LQR2} был разработан регулятор на основе наблюдателя с использованием линейно--квадратичного регулятора для систем с различными возмущениями. Этот подход широко применим к классу задач оптимального управления со структурными ограничениями. 

{\objectsubject} 

Объектом исследования является оптимальное робастное управление шагающими роботами с мультипликативными и аддитивными неопределённостями, полученными при линеаризации динамической системы.

Предметом исследования являются методы формулирования линейных матричных неравенств для нахождения коэффициентов регулятора и наблюдателя с последующими численными вычислениями для модели шагающего робота.

{\aimtasks} 

Целью диссертационной работы является разработка ортогональных методов оптимального робастного управления шагающими роботами с использованием матричных неравенств.

Для~достижения поставленной цели необходимо было решить следующие задачи:
\begin{enumerate}[beginpenalty=10000] % https://tex.stackexchange.com/a/476052/104425
	\item анализ текущих методов робастного управления для шагающих роботов, вариантов формулирования математической модели для описания движения шагающего робота, способов задания неопределённостей в модели;
	\item разработка метода нахождения оптимального робастного управления для системы с мультипликативными неопределённостями, описывающую шагающего робота;
	\item разработка метода нахождения оптимального робастного управления для системы с мультипликативными и аддитивными неопределённостями, описывающую шагающего робота;
	\item разработка метода нахождения оптимального робастного управления для системы с аддитивными неопределённостями и динамическим управлением, описывающую шагающего робота;
	\item численный анализ разработанных линейных матричных неравенств для нахождения коэффициентов регулятора и наблюдателя для робастного управления, решение прямой задачи динамики, исследование зависимостей результатов от параметров оптимизации;
	\item подтверждение верность разработанных методов на математической модели шагающего робота.
\end{enumerate}

{\methods} 

Для решения вышеуказанных задач и достижения цели использовались методы теории управления и устойчивости, дифференциальных уравнений, численные-аналитические, выпуклой оптимизации, методы математического моделирования, эмпирические данные.

{\defpositions}

\begin{enumerate}[beginpenalty=10000] % https://tex.stackexchange.com/a/476052/104425
	\item метод нахождения оптимального робастного управления для системы с мультипликативными неопределённостями и обратной связью по полному состоянию. Позволяет находить оптимальные коэффициенты регулятора для различных неопределённостей, удовлетворяющих заданным условиям и с использованием линейных матричных неравенств; 
	\item метод нахождения оптимального робастного управления для системы с аддитивными неопределённостями и с динамической обратной связью. Позволяет находить оптимальные коэффициенты регулятора для различных неопределённостей, удовлетворяющих заданным условиям и с использованием линейных матричных неравенств; 
	\item метод численного анализа для проведения численных экспериментов и анализ их результатов, которые верифицировали предыдущие положения.
\end{enumerate}

{\compliances} 

Содержание диссертации соответствует пунктам паспорта научной специальности 1.2.2 --- «Математическое моделирование, численные методы и комплексы программ» (физико--математические науки).
\begin{enumerate}[beginpenalty=10000]
	\item математическое моделирование
	\item численные методы
	\item комплексы программ
\end{enumerate}

{\novelty}

Основные результаты, полученные в работе, являются новыми. В частности
\begin{enumerate}[beginpenalty=10000] % https://tex.stackexchange.com/a/476052/104425
	\item впервые разработаны методы робастного оптимального управления с обратной связью по полному состоянию, используя линейные матричные неравенства для систем с механическими ограничениями и неопределённостями;
	\item впервые разработаны методы робастного оптимального управления с динамической обратной связью, используя линейные матричные неравенства для систем с механическими ограничениями и неопределённостями;
	\item было проведено оригинальное исследование модели шагающего робота, её ортогональной декомпозиции и линеаризации перед разработкой оптимального робастного управления.
\end{enumerate}

{\probation}

Достоверность полученных результатов обеспечивается строгими математическими доказательствами теоретических результатов, корректностью исходных и упрощающих допущений для численного вычисления, адекватными результатами вычислительных экспериментов, использованием уравнений, методов и подходов, строго обоснованных в научной литературе, апробированы и хорошо зарекомендованных при разработке робастного управления для систем с неопределённостями. 

{\influence} 

Теоретическая значимость состоит в разработке нового метода для вычисления коэффициентов регулятора и наблюдателя для робастного управления системы с мультипликативными неопределённостями с использованием линейных матричных неравенств.

Практическая значимость результатов подтверждается применением результатов теоретической части в практике: приводятся формулировки оптимизационных проблем для поиска коэффициентов регуляторов и наблюдателей для управления динамических систем, описывающих динамику шагающего робота. 

{\reliability} 

Результаты, включённые в данную работу, опубликованы в 2 печатных изданиях, 1 из которых изданы в журналах, рекомендованных ВАК, 1 --- в периодических научных журналах, индексируемых \textit{Web of Science} и \textit{Scopus}. 

Результаты находятся в соответствии с результатами, полученными другими авторами. Результаты работы неоднократно докладывались на научно--методических семинарах и были использованы в учебной программе Университета Иннополис.

{\contribution} 

Все результаты, выносимые на защиту, получены лично автором: выбор методик решения задач, разработка алгоритма построения оптимального робастного управления, написание программ для моделирования шагающих роботов и использования разработанных численных экспериментов и анализ результатов расчётов. Постановка задач, обсуждение и интерпретация полученных результатов осуществлялась совместно с научным руководителем. % Характеристика работы по структуре во введении и в автореферате не отличается (ГОСТ Р 7.0.11, пункты 5.3.1 и 9.2.1), потому её загружаем из одного и того же внешнего файла, предварительно задав форму выделения некоторым параметрам

\textbf{Объем и структура работы} 

Диссертация состоит из~введения,
\formbytotal{totalchapter}{глав}{ы}{}{},
заключения и
\formbytotal{totalappendix}{приложен}{ия}{ий}{}.
%% на случай ошибок оставляю исходный кусок на месте, закомментированным
%Полный объём диссертации составляет  \ref*{TotPages}~страницу
%с~\totalfigures{}~рисунками и~\totaltables{}~таблицами. Список литературы
%содержит \total{citenum}~наименований.
%
Полный объём диссертации составляет
\formbytotal{TotPages}{страниц}{у}{ы}{}, включая
\formbytotal{totalcount@figure}{рисун}{ок}{ка}{ков} и
\formbytotal{totalcount@table}{таблиц}{у}{ы}{}.
Список литературы содержит
\formbytotal{citenum}{наименован}{ие}{ия}{ий}.

Во введении приводятся обоснование актуальности темы исследования, степень разработанности темы исследования, формулируются объект и предмет исследования, цель, задачи, научная новизна, методы исследования, положения, выносимые на защиту, описываются теоретическая и практическая значимость работы, внедрение результатов, степень достоверности полученных результатов; приводятся сведения об апробации работы и личном вкладе автора.

В первой главе проведён литературный обзор по теме исследования. Изучены особенности шагающих роботов, а также методы построения управления для них. Исследованы причины для ортогональной декомпозиции и различные методы для её применения. Подняты вопросы робастного управления, в том числе для шагающих роботов. Исследована мультипликативная неопределённости в модели. Приведены методы основанные на линейных матричных неравенствах для робастного управления. Описаны используемые математические инструменты.

Во второй главе приведено описание шагающих роботов и их особенностей на примере четырёхногих роботов. Показан алгоритм линеаризации динамической модели и неопределённостей.

В третей главе представлены результаты разработки оптимизационной проблемы для поиска коэффициентов регулятора и наблюдателя для системы с мультипликативной неопределённостью. Приведены результаты разработки алгоритма для поиска коэффициентов регулятора по обратной связи.

В четвёртой главе диссертационной работы представлены результаты исследования численных методов приведённых в предыдущей главе и проведены численные эксперименты на модели четвероногого робота.

В заключении перечислены результаты работы и направления дальнейших исследований.