\section{Математические инструменты}\label{sec:ch3/sect1}
В данном разделе мы обобщим несколько результатов, которые будут использованы в последующих разделах работы.
\begin{lemma}\label{lemma:Young}
	(Неравенство Юнга).
	Существует множество различных версий неравенства Юнга, которые используются для работы с линейно матричными неравенствами. Все они могут быть выведены друг из друга. В данной работе мы будем использовать следующие две версии:
	%
	Для любой положительно определённой матрицы $M>0$, положительных скаляров $\epsilon > 0, \nu > 0$ и матриц ${X}, {Y}, {F}$, где ${F}\T{F}\leq \nu{I}$, следующие неравенства верны \cite{BOYED1994}:
	%
	\begin{align}
		\label{eq:Young_relation_BMI}
		{X}\T{Y} + {Y}\T{X}  \leq {X}\T 
		\epsilon {M}^{-1}{X} + \frac{1}{\epsilon}   {Y}\T  {M}{Y}, 
		\\
		\label{eq:Young_robust_2}
		{X}\T{F}{Y} + {Y}\T{F}\T{X}  \leq \epsilon {X}\T{X} +  \frac{\nu}{\epsilon} {Y}\T {Y}.
	\end{align}
	%
	следуя \cite{LIEN2008} мы можем выбрать $\nu=\epsilon^2$ для формулирования следующего неравенства:
	%
	\begin{equation}
		\label{eq:updated_Young_robust_2}
		{X}^T{F}{Y} + {Y}\T{F}\T{X}  \leq \epsilon {X}\T{X} + \epsilon {Y}\T{Y}.
	\end{equation}
\end{lemma}

\begin{lemma}\label{lemma:S_procedure}
	(S-Процедура).
	Рассмотрим ${M},{N} \in \mathbb{R}^{n\times n}$. Если существует скаляр $\gamma>0$, который отвечает следующему условию ${M}+\gamma {N}<0$, тогда $x\T {N} x\geq 0$ подразумевает $x\T{M}x\leq 0$ \cite{BOYED1994}.
\end{lemma}

\begin{lemma}\label{lemma:Schur}
	(Дополнение Шура \cite{Schur}).
	Для любой симметричной матрицы ${A}\in \mathbb{S}^n$ и ${C}\in \mathbb{S}^m$ и матрицы ${B}\in \mathbb{R}^{n\times m}$, а также их конкатенации:
	\noindent \begin{align*}
		{M}= \begin{bmatrix}
			{A} & {B} \\
			{B}\T & {C} 
		\end{bmatrix},
	\end{align*}
	%
	Следующие утверждения равнозначны:
	% 
	\noindent
	\begin{enumerate}
		\item ${M} < 0$,
		\item ${A}-{B}{C}^{-1}{B}\T < 0 , {C}< 0$.
		\item ${C}-{B}\T{A}^{-1}{B}< 0 , {A}< 0$
	\end{enumerate}
\end{lemma}

\section{Предлагаемый метод}\label{sec:ch3/sect2}
\section{Предлагаемый метод}\label{sec:ch3/sect2}
\subsection{Аддитивная и мультипликативная неопределённость}\label{sec:ch3/sect2/sub1}
Рассмотрим следующую динамическую систему:
%
\begin{equation}
	\label{eq:part1_linear_dynamics}
	\begin{cases}
		\dot z=({A}_n+\Delta {A}_n)z + {A}_r\zeta + {B}u,\\
		y={C}{N}z+{C}{R}\zeta,
	\end{cases}
\end{equation}
%
где $z$ это динамические состояния, $\zeta = \text{const}$ - статические состояния и $\Delta {A}_n$ представляет мультипликативную модель неопределенностей и имеет следующую структуру:
%
\begin{equation}
	\label{eq:part1_uncertainty}
	\Delta {A}_n={M}_1{F}_1{N}_1 \quad \text{и} \quad {F}_1\T{F}_1\leq \nu {I},
\end{equation}
%
где ${M_1} \in \mathbb{R}^{n_z \times d}$ и 
${N_1} \in \mathbb{R}^{d \times n_z}$ известные матрицы, ${F}_1$ - неизвестная матрица ограниченная по норме и $\nu$ - неизвестный радиус - скаляр, определяющий лимит наложенный на норму ${F}_1$.

Следуя \cite{SAVIN2021} мы можем ввести наблюдатель Люнберга:
%
\begin{equation}
	\begin{bmatrix}
		\dot{\hat{z}} \\
		\dot{\hat{\zeta}}
	\end{bmatrix}=\begin{bmatrix}
		{A}_n & {A}_r \\
		0 & 0
	\end{bmatrix}
	\begin{bmatrix}
		\hat{z}\\ \hat{\zeta}
	\end{bmatrix}
	+  \begin{bmatrix}
		{B}\\0
	\end{bmatrix}u + {L} \left( y-\begin{bmatrix}
		{C}{N} & {C}{R}
	\end{bmatrix} \begin{bmatrix}
		\hat{z}\\ \hat{\zeta}
	\end{bmatrix} \right),
\end{equation}
%
где ${L}$ коэффициент наблюдателя.

Определим ошибку оценки состояния как $e = [ (z-\hat{z})\T \ \ (\zeta-\hat{\zeta})\T ]\T$ и введем блочную матрицу:
${S} = \begin{bmatrix}
	{I} \\ 0
\end{bmatrix}$, 
$\bo{E}=[ \bo{N} \ \ \bo{R}]$, и 
$
\bo{A}_c=    \begin{bmatrix}
	\bo{A}_r  & \bo{A}_{\rho} \\
	0  & 0
\end{bmatrix}
$
записываем ошибку наблюдателя динамики:
%
\begin{equation}
	\label{eq:part1_error_dynamics}
	\dot e= (\bo{A}_e-\bo{L}\bo{C}\bo{E})e +\bo{S}\Delta \bo{A}_n z.
\end{equation}
%
Вводим закон управления:
%
\begin{equation}
	u=\bo{K}_z \hat{z}+\bo{K}_{\zeta} \hat{\zeta},
\end{equation}
%
и определяем $\bo{K}=\begin{bmatrix}
	\bo{K}_z & \bo{K}_{\zeta}
\end{bmatrix}$ записываем динамическую систему с обратной связью:
%
\begin{equation}
	\label{eq:part1_active_dynamics}
	\dot{z}=(\bo{A}_n+\Delta \bo{A}_n +\bo{B}\bo{K}_z)z-\bo{B}\bo{K}e+(\bo{A}_r+\bo{B}\bo{K}_{\zeta})\zeta.
\end{equation}
%
Коэффициент регулятора $\bo{K}_{\zeta}$ может быть выбран как:
%
\begin{equation}
	\label{eq:part1_static_control}
	\bo{K}_{\zeta}=-\bo{B}^{\dagger}\bo{A}_r.
\end{equation}
%
Пока столбцы$\bo{A}_r$ лежат в подпространстве столбцов $\bo{B}$ и существует точная оценка состояния, данный закон управления сводит на нет эффект $\zeta$ на динамику. В данном случае,отмечая что $\bo{K}_z=\bo{K}\bo{S}$,  мы можем представить ошибку наблюдателя и динамику робота как систему уравнений:
%
\begin{equation}
	\label{eq:part1_system}
	\begin{bmatrix}
		\dot{z} \\ \dot{e}
	\end{bmatrix}=\begin{bmatrix}
		(\bo{A}_n+\Delta \bo{A}_n +\bo{B}\bo{K}\bo{S}) & -\bo{B}\bo{K} \\
		\bo{S} \Delta \bo{A}_n & (\bo{A}_e-\bo{L}\bo{C}\bo{E})        \end{bmatrix}\begin{bmatrix}
		z \\ e
	\end{bmatrix}.
\end{equation}
%
Рассмотрим проблему нахождения таких коэффициентов регулятора и наблюдателя, которые будут давать устойчивую систему для всех допустимых значений $\Delta \bo{A}_n$.
\clearpage

\clearpage
