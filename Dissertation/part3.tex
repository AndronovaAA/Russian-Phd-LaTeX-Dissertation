\chapter{Предлагаемый метод}\label{ch:ch3}

\section{Математические инструменты}\label{sec:ch3/sect1}
В данном разделе мы обобщим несколько результатов, которые будут использованы в последующих разделах работы.
\begin{lemma}\label{lemma:Young}
	(Неравенство Юнга).
	Существует множество различных версий неравенства Юнга, которые используются для работы с линейно матричными неравенствами. Все они могут быть выведены друг из друга. В данной работе мы будем использовать следующие две версии:
	%
	Для любой положительно определённой матрицы $M>0$, положительных скаляров $\epsilon > 0, \nu > 0$ и матриц ${X}, {Y}, {F}$, где ${F}\T{F}\leq \nu{I}$, следующие неравенства верны \cite{BOYED1994}:
	%
	\begin{align}
		\label{eq:Young_relation_BMI}
		{X}\T{Y} + {Y}\T{X}  \leq {X}\T 
		\epsilon {M}^{-1}{X} + \frac{1}{\epsilon}   {Y}\T  {M}{Y}, 
		\\
		\label{eq:Young_robust_2}
		{X}\T{F}{Y} + {Y}\T{F}\T{X}  \leq \epsilon {X}\T{X} +  \frac{\nu}{\epsilon} {Y}\T {Y}.
	\end{align}
	%
	следуя \cite{LIEN2008} мы можем выбрать $\nu=\epsilon^2$ для формулирования следующего неравенства:
	%
	\begin{equation}
		\label{eq:updated_Young_robust_2}
		{X}^T{F}{Y} + {Y}\T{F}\T{X}  \leq \epsilon {X}\T{X} + \epsilon {Y}\T{Y}.
	\end{equation}
\end{lemma}

\begin{lemma}\label{lemma:S_procedure}
	(S-Процедура).
	Рассмотрим ${M},{N} \in \mathbb{R}^{n\times n}$. Если существует скаляр $\gamma>0$, который отвечает следующему условию ${M}+\gamma {N}<0$, тогда $x\T {N} x\geq 0$ подразумевает $x\T{M}x\leq 0$ \cite{BOYED1994}.
\end{lemma}

\begin{lemma}\label{lemma:Schur}
	(Дополнение Шура \cite{Schur}).
	Для любой симметричной матрицы ${A}\in \mathbb{S}^n$ и ${C}\in \mathbb{S}^m$ и матрицы ${B}\in \mathbb{R}^{n\times m}$, а также их конкатенации:
	\noindent \begin{align*}
		{M}= \begin{bmatrix}
			{A} & {B} \\
			{B}\T & {C} 
		\end{bmatrix},
	\end{align*}
	%
	Следующие утверждения равнозначны:
	% 
	\noindent
	\begin{enumerate}
		\item ${M} < 0$,
		\item ${A}-{B}{C}^{-1}{B}\T < 0 , {C}< 0$.
		\item ${C}-{B}\T{A}^{-1}{B}< 0 , {A}< 0$
	\end{enumerate}
\end{lemma}

\section{Предлагаемый метод}\label{sec:ch3/sect2}
\clearpage
