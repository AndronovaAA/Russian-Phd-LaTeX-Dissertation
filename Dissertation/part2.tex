\chapter{Метод математического моделирования для построения управления шагающих роботов}\label{ch:ch2}

\section{Математический метод описания модели и механики контакта}\label{sec:ch2/sect2}
Для построения математической модели робота необходимо учесть механику контакта робота с окружающей средой.
Одним из способов описания контакта является метод, используемый для описания движения с помощью уравнений Лагранжа. В дополнение к классическим формулировкам задач с граничными условиями механики сплошной среды, в контактных задачах вводятся специальные граничные условия, которые управляют движением граничных поверхностей и потенциальных сингулярностей. Для классических контактных задач эти условия выражают ограничение на отсутствие проникновения, закон действия и реакции Ньютона и закон поверхностного трения. Нормальные проекции этих условий предотвращают взаимное проникновение несмешивающихся сред, а тангенциальные проекции учитывают силы трения между соприкасающимися поверхностями \cite{Shamim2024}.

Рассмотрим односторонний точечный контакт --- ограничение ${r}({q}) = 0$, где ${r} \in \mathbb{R}^3$, с соответствующим якобианом ограничений ${J} = \frac{\partial {r}}{\partial {q}}$ и силой реакции ${f} \in \mathbb{R}^3$ \cite{Posa2014}.
%
Сила реакции лежит в конусе трения ${f} \in \mathcal{C}$, с нормальным направлением ${n}$ и коэффициентом трения $\mu$.
%
Такой контакт (ограничение) называется односторонним, потому что связанная с ним сила реакции может «давить», но не «тянуть» на поверхность опоры.

В методах Лагранжа граничные скорости между жёсткими и деформируемыми телами либо задаются, либо корректируются на основе проникновения узлов сетки в запрещённые области, представляющие жёсткие тела. Нормальные компоненты скорости и смещения корректируются для предотвращения проникновения. Если во время коррекции прикладываются внешние нормальные нагрузки, можно определить контактное давление, а трение обрабатывается аналогично другим алгоритмам контакта.

\section{Манипуляторные уравнения для описания динамики шагающих роботов}\label{sec:ch2/sect3}

Рассмотрим уравнение Лагранжа:
%
\begin{equation}
	\frac{d}{dt} \bigg( 
	\frac{\partial T }{\partial \dot{{q}}}
	\bigg) - 
	\frac{\partial T }{\partial {q}} = \tau,
\end{equation}
%
где $T$ --- кинетическая энергия, ${q}$ --- вектор обобщённых координат, а $\tau$ --- обобщённые крутящие моменты. Заметим, что кинетическая энергия может быть описана как $T = \frac{1}{2} \dot{{q}}\T {H} \dot{{q}}$, где ${H}$ --- обобщённая матрица инерции. Матрица ${H}$ является симметричной, положительно определённой и полного ранга.

Обобщённые силы $\tau$ могут быть сформированы декартовыми силами или декартовыми крутящими моментами. Опишем соотношения между декартовой силой ${f}$ и связанной с ней обобщённой силой:

\begin{equation}
	\label{eq:part2_tau}
	\tau_i = \left( \frac{\partial {r}_i}{\partial {q}} \right)\T {f}_i,
\end{equation}
%
где ${r}_i = {r}_i({q})$ --- вектор, описывающий положение точки приложения силы ${f}_i$, как функцию обобщённых координат ${q}$.

При определении якобиана как ${J}_i = \frac{\partial {r}_i}{\partial {q}}$, соотношение \eqref{eq:part2_tau} будет выглядеть как:
%
\begin{equation}
	\tau_i = ({J}^r_i)\T {f}_i.
\end{equation}
%
Соотношение между моментом ${m}$ и связанной с ним обобщённой силой:

\begin{equation}
	\label{eq:part2_tau_j}
	\tau_i = \left( \frac{\partial \omega_i}{\partial \dot{{q}}} \right)\T {m}_i,
\end{equation}
%
где $\omega_i = \omega_i({q}, \dot{{q}})$ --- угловая скорость тела, к которому приложен момент ${m}$.
Определяя якобиан как ${J}^\omega_i = \frac{\partial \omega_i}{\partial \dot{{q}}}$, выражение \eqref{eq:part2_tau_j} принимает вид:
%
\begin{equation}
\tau_i = ({J}^\omega_i)\T {m}_i.
\end{equation}
Определим крутящий момент как ${w} = \begin{bmatrix}
	{f} \\\ {m}
\end{bmatrix}$ и выразим обобщённую силу:
%
\begin{equation}
	\tau_i = \begin{bmatrix}
		{J}^r_i \\\ {J}^\omega_i
	\end{bmatrix}\T
	{w}_i
	=
	{J}_i\T {w}_i.
\end{equation}
Заметим, что полная обобщённая сила может быть вычислена как:
%
\begin{equation}
	\tau = \sum {J}_i\T {w}_i.
\end{equation}
%
Уравнения Лагранжа с ограничениями имеют вид:
%
\begin{equation}
	\begin{cases}
		\frac{d}{dt} \bigg( 
		\frac{\partial T }{\partial \dot{{q}}}
		\bigg) - 
		\frac{\partial T }{\partial {q}} = \tau + \left( \frac{\partial {r}}{\partial {q}} \right)\T \lambda,
		\\
		{r}({q}) = 0,
	\end{cases}
\end{equation}
%
\nomenclature{\(\T\)}{Транспонирование матрицы}
%
где ${r}({q}) = 0$ --- ограничения, а $\lambda$ --- силы реакции. Считаем $\lambda$ конкатенацией всех сил реакции, связанных с ограничениями.

Уравнения манипулятора записываются следующим образом:
%
\begin{equation}
	{H} \ddot{{q}} + {C} \dot{{q}} + {g} = \tau,
\end{equation}
%
где ${H}={H}({q})$ --- обобщённая матрица инерции, ${C}\dot{{q}}$ --- обобщённые инерционные силы, ${g}={g}({q})$ --- обобщённые гравитационные силы.
Уравнения манипулятора запишем как:
%
\begin{equation}
	\begin{cases}
		{H} \ddot{{q}} + {C} \dot{{q}} + {g} = \tau + \left( \frac{\partial {r}}{\partial {q}} \right)\T \lambda,
		\\
		{r}({q}) = 0.
	\end{cases}
\end{equation}
%
Определив якобиан ограничений ${J} = \frac{\partial {r}}{\partial {q}}$:
%
\begin{equation}
	\begin{cases}
		{H} \ddot{{q}} + {C} \dot{{q}} + {g} = \tau + {J}\T \lambda,
		\\
		{r}({q}) = 0.
	\end{cases}
\end{equation}
Дифференцируя ограничение ${r}({q})$, получаем:
%
\begin{align}
	\frac{\partial {r}}{\partial {q}} \frac{\partial {q}}{\partial t} = 0,\\
	{J} \dot{{q}} = 0.
\end{align}
%
Продифференцировав ещё раз, получим:
%
\begin{equation}
	{J} \ddot{{q}} + \dot{{J}} \dot{{q}} = 0.
\end{equation}
%
Заменяя ограничения их вторыми производными:
%
\begin{equation}
	\begin{cases}
		{H} \ddot{{q}} + {C} \dot{{q}} + {g} = \tau + {J}\T \lambda,
		\\
		{J} \dot{{q}} + \dot{{J}} \dot{{q}} = 0.
	\end{cases}
\end{equation}
%
Это ДАУ в переменных ${q}$ и $\lambda$. Перепишем систему уравнений в векторно--матричной форме:
%
\begin{equation}
	\begin{bmatrix}
		{H} & -{J}\T \\
		{J} & 0
	\end{bmatrix}
	\begin{bmatrix}
		\ddot{{q}} \\
		\lambda
	\end{bmatrix}
	=
	\begin{bmatrix}
		\tau - {C} \dot{{q}} - {g} \\
		-\dot{{J}} \dot{{q}}
	\end{bmatrix}.
\end{equation}

Приведённое выше матрично-векторное уравнение может быть решено при условии, что дополнение Шура ${J}\T {H}^{-1}{J}$ имеет полный ранг:

\begin{equation}
	\begin{bmatrix}
		\ddot{{q}} \\
		\lambda
	\end{bmatrix}
	=
	\begin{bmatrix}
		{H}^{-1}-{H}^{-1} {J}\T {H}_J {J} {H}^{-1} &
		{H}^{-1} {J}\T {H}_J \\
		-{H}_J {J} {H}^{-1} & {H}_J
	\end{bmatrix}
	\begin{bmatrix}
		\tau - {C} \dot{{q}} - {g} \\
		-\dot{{J}} \dot{{q}}
	\end{bmatrix},
\end{equation}
%
где ${H}_J = ( {J} {H}^{-1} {J}\T )^{-1}$.
%
Решение можно записать:
%
\begin{align}
	\ddot{{q}}
	&=
	({I}-{H}^{-1} {J}\T {H}_J {J} ) {H}^{-1}(\tau - {C} \dot{{q}} - {g}) 
	-
	{H}^{-1} {J}\T {H}_J \dot{{J}} \dot{{q}},
	\\
	\lambda
	&=
	-{H}_J {J} {H}^{-1} (\tau - {C} \dot{{q}} - {g})
	-
	{H}_J \dot{{J}} \dot{{q}}.
\end{align}
%
Подставляя выражение для $\lambda$ в динамическое уравнение ${H}\ddot{{q}} - {J}\T\lambda
=
-{h}$, где ${h} = {C} \dot{{q}} + {g} - \tau$ получаем:
\begin{align}
	{H}\ddot{{q}} + 
	{J}\T{H}_J {J} {H}^{-1} (-{h}) +
	{J}\T {H}_J \dot{{J}} \dot{{q}}
	=
	(-{h}),
	\\
	{{H}{{q}} + 
		({I}-
		{J}\T{H}_J {J} {H}^{-1}) {h} +
		{J}\T {H}_J \dot{{J}} \dot{{q}}
		=
		0}.
\end{align}
%
Рассмотрим следующее уравнение:
%
\begin{align}
	({I}-
	{J}\T{H}_J {J} {H}^{-1})({H}\ddot{{q}}+{h})=0,
	\\
	{H}\ddot{{q}} - {J}\T{H}_J {J} \ddot{{q}}
	+
	({I}-
	{J}\T{H}_J {J} {H}^{-1}){h} = 0.
\end{align}
%
Используя равенство ${J} \ddot{{q}} = -\dot{{J}} \dot{{q}}$:
%
\begin{align}{
		{H}\ddot{{q}} + ({I}-{J}\T{H}_J {J} {H}^{-1}){h} + {J}\T{H}_J \dot{{J}} \dot{{q}} = 0}.
\end{align}
%
Таким образом, динамика может быть переписана в виде: 
%
\begin{align}
	{P}_S({H}\ddot{{q}}+{h})=0,
\end{align}
%
где ${P}_S = {I}-{J}\T{H}_J {J} {H}^{-1} = 
{I}-{J}\T ( {J} {H}^{-1} {J}\T )^{-1} {J} {H}^{-1}$ --- это проектор Шура.

Проектор ${P}_S$ не меняет количество уравнений, но он проецирует динамику на ограничения, это позволяет отказаться от сил реакции.
%
\section{Формулировка обратной задачи кинематики для шагающих роботов}\label{sec:ch2/sect4}

Обратная кинематика для системы $\mathcal{G}$ --- это задача нахождения управляющего входа ${u}$ при заданном состоянии $({q}, \dot{{q}})$ и желаемом значении производной высшего порядка $\ddot{{q}}$ \cite{Righetti2011}. При решении обратной задачи кинематики можно сформировать входные сигналы для управлением движения.
\begin{comment}
Динамика свободно плавающего робота с жёстким телом, подверженного внешним ограничениям, в общем случае задаётся следующим образом
\begin{equation}
	M \ddot{q} +h =S\T\tau + J_c\T \lambda,
\end{equation}
при $k$ ограничениях
\begin{equation}
	J_c \ddot{q} = b(q,\dot{q}),
\end{equation}
где $M \in \mathbb{R}^{n+6 \times n+6}$ - матрица инерции динамики жёсткого тела, $h \in \mathbb{R}^{n+6}$ - обобщённый вектор силы, содержащий эффекты Кориолиса, центробежные и гравитационные эффекты, $\tau \in \mathbb{R}^{n}$ - вектор воздействия и $S \in \mathbb{R}^{n \times n+6}$ - матрица выбора соединений, отражающая недостаточность действия - например, для большинства роботов с плавающей базой S будет матрицей тождества в первой подматрице $n \times n$, а в остальных местах - нули. 
$J_c \in \mathbb{R}^{k \times n+6}$ - Якобиан $k$ ограничений c $\lambda \in \mathbb{R}^{k}$ множителями Лагранжа, соответствующих ограничивающим силам.
\end{comment}
Предположим, что ${h} = {C}\dot{{q}} + {g} - {T}{u}$, где ${u}$ --- вектор моментов двигателя, ${T}$ --- матрица управления, а ${T}{u}$ --- обобщённые силы, создаваемые моментами двигателя:
%
\begin{align}
	{P}_S({H}\ddot{{q}} + {C}\dot{{q}}+ {g}) = {P}_S{T}{u}.
\end{align}
%
Учитывая динамику в виде ${P}_S({H}\ddot{{q}}^* + {C}\dot{{q}}+ {g}) = {P}_S{T}{u}$, где $\ddot{{q}}^*$ --- желаемое ускорение, решение наименьших квадратов для ${u}$, используя псевдообращение:
%
\begin{align}
	{u}
	=
	({P}_S{T})^+{P}_S({H}\ddot{{q}}^* + {C}\dot{{q}}+ {g}).
\end{align}
%
Подставляя выражение для ${u}$ в исходную динамику:
%
\begin{align}
	{P}_S({H}\ddot{{q}}^* + {C}\dot{{q}}+ {g}) = {P}_S{T}({P}_S{T})^+{P}_S({H}\ddot{{q}} + {C}\dot{{q}}+ {g}),
	\\
	({I}-{P}_S{T}({P}_S{T})^+){P}_S({H}\ddot{{q}}^* + {C}\dot{{q}}+ {g}) = 0.
\end{align}
%
\nomenclature{\(^{-1}\)}{Обратная матрица}
\nomenclature{\(^+\)}{Псевдообратная матрица}
\nomenclature{\(*\)}{Обозначает элементы симметричной матрицы}
\nomenclature{\(\text{diag()}\)}{Квадратная матрица с входящим вектором в главной диагонали}
\nomenclature{\(\text{tr()}\)}{След матрицы}
%
Равенство выполняется, если ${P}_S({H}\ddot{{q}}^* + {C}\dot{{q}}+ {g})$ лежит в пространстве столбцов матрицы $({P}_S{T})$.
Если равенство выше не выполняется, то такое ускорение $\ddot{{q}}^*$ недостижимо.

\section{Формулировка прямой задачи кинематики и математическое моделирование шагающих роботов}\label{sec:ch2/sect5}
Прямая динамика для системы $\mathcal{G}$ --- это задача нахождения производной высшего порядка $\ddot{{q}}$ при заданном состоянии $({q}, \dot{{q}})$ и управляющем входе ${u}$. При решении задач прямой кинематики мы можем рассматривать контакт разными способами.

Существует несколько основных сценариев того, как сила трения может быть нарушена:
%
\begin{enumerate}
	\item Сила трения (касательная составляющая силы реакции) может лежать на границе конуса трения, что приводит к скольжению. Длина силы трения равна $\mu {n}\T {f}$, движение по касательной становится возможным. Как правило, следует избегать построения сложных схем управления на основе моделей, требующих скольжения.
	\item Нормальная сила может стать равной нулю, что позволяет разорвать контакт.
\end{enumerate}

Разрыв контакта происходит, когда нормальная реакция становится равной нулю, а точка контакта приобретает ускорение в проекции на нормаль к поверхности.
%
Для описания процесса приобретения и разрыва контакта (локально) мы можем использовать ограничение дополняемости:
%
\begin{align}
	\label{eq:part2_cond}
	\begin{cases}
		f_n r_n = 0, \\
		f_n \geq 0, \\
		r_n \geq 0,
	\end{cases}
\end{align}
где $f_n = {n}\T{f}$ и $r_n ={n}\T{r}$. Мы можем описать это ограничение следующим образом: либо $f_n = 0$ и тогда $r_n$ может принимать любое положительное число, то есть контакт может быть разорван; либо $r_n = 0$, контакт сохраняется и, следовательно, нормальная сила реакции положительна.

Рассмотрим случай, когда полная сила реакции в $i$--точке контакта равна нормальной силе реакции: ${f}_i = {f}_{n,i} = f_{n,i} {n}_i$. Тогда обобщённое ограничение $\phi$ строится путём конкатенации:
%
\begin{align}
	\phi = \begin{bmatrix}
		{n}_1\T \left({r}_1({q}) - {r}_1^0 \right) \\
		... \\
		{n}_m\T \left({r}_m({q}) - {r}_m^0 \right)
	\end{bmatrix}
	= 0.
\end{align}
%
В этом случае ограничение дополнительности становится:
%
\begin{align}
	\begin{cases}
		\phi\T \lambda = 0, \\
		\phi \geq 0, \\
		\lambda \geq 0.
	\end{cases}
\end{align}
%
Как мы видим, эта формулировка включает в себя все ограничения (все точки контакта) одновременно, и может быть непосредственно использована в сочетании с уравнениями манипулятора.

Контактные ограничения, сформулированные с помощью условий дополнительности, естественным образом вписываются в прямую формулировку траекторной оптимизации. Вместо того чтобы решать линейные задачи дополнения для контактных сил $\lambda$ на каждом шаге, мы напрямую оптимизируем пространство выполнимых состояний, управляющих входов, ограничивающих сил и длительности траектории. 

Возникновение контакта может означать быстрое (математически --- мгновенное) изменение импульса (обобщённого момента) механической системы. Это называется столкновением.
%
Простые модели столкновения включают неупругое столкновение (часть энергии теряется при столкновении) и упругое столкновение (энергия сохраняется).
%
Для одномерной точечной массы неупругое столкновение описывается с помощью коэффициента восстановления $\xi$:
%
\begin{equation}
	v^+ = -\xi v^-,
\end{equation}
%
где $v^-$ и $v^+$ --- скорости точечной массы до и после столкновения.

Следует отметить, что ограничение \eqref{eq:part2_cond} может быть трудно правильно сформулировать в случае, когда непроникающее конфигурационное пространство невыпукло, хотя в этой области была проделана работа по моделированию \cite{Nguyen2010}. Стандартные вариации этой формулировки могут быть сделаны так, чтобы также учитывать фрикционный контакт \cite{Tassa2012, brogliato2012nonsmooth}.

Рассмотрим контакт как виртуальные пружины.
Модель представляет собой точечную массу $m$, на безмассовой пружинящей ноге с длиной покоя $l_0$, и коэффициент пружины $k$. Конфигурация системы задаётся положением центра масс $x, z$, а также длиной $l$, и углом ноги $\theta$. Динамика моделируется по частям --- одна динамика управляет фазой полёта, а другая --- фазой стойки. На самом деле мы используем разные переменные состояния для каждой фазы динамики. В фазе полёта мы используем переменные состояния: $\mathbf{x} = [x,z,\dot{x},\dot{z}]^T$. Динамика в этой фазе:
\begin{equation}
	\dot{\mathbf{x}} =
	\begin{bmatrix} \dot{x} \\ \dot{z} \\ 0 \\ - g \end{bmatrix}.
\end{equation}
Поскольку нога безмассовая, мы предполагаем, что ею можно мгновенно управлять в любом положении, поэтому разумно взять угол наклона ноги в качестве управляющего воздействия $\theta = u$.

Во время «фазы стойки» мы пишем динамику в полярных координатах, с ногой, закрепленной в начале координат. Используя переменные состояния $\mathbf{x} = [r, \theta, \dot{r}, \dot\theta]^T,$ 
местоположение массы определяется $$\mathbf{x}_m = \begin{bmatrix} - r \sin\theta \\\ r \cos\theta \end{bmatrix}. $$ Полная энергия складывается из кинетической энергии (массы) и потенциальной энергии силы тяжести и пружины ноги: $$T = \frac{m}{2} (\dot{r}^2 + r^2 \dot\theta^2 ), \quad U = mgr\cos\theta + \frac{k}{2}(l_0 - r)^2. $$ Подставляя их в уравнение Лагранжа, получаем динамику стойки: 
\begin{gather*} 
	m \ddot{r} - m r \dot\theta^2 + m g \cos\theta - k (l_0 - r) = 0, \\\ 
	m r^2 \ddot{\theta} + 2mr\dot{r}\dot\theta - mgr \sin\theta = 0.
\end{gather*} 
Мы предполагаем, что фаза старта полностью баллистическая, в этой фазе нет никаких управляющих воздействий.

Идеализация пружинящей ноги означает, что при столкновении с землёй энергия не теряется. Столкновение приводит к переходу в другое пространство состояний и к другой модели динамики, но без мгновенного изменения скорости. Переход от полёта к стойке происходит, когда $z -
l_0\cos\theta \le 0.$  Переход от стойки к полёту происходит, когда нога, стоящая в стойке, достигает длины покоя $r \ge l_0$.

\section{Метода робастного управления шагающими роботами}\label{sec:ch2/sect6}

Управление шагающими роботами представляет собой сложную задачу, которая решается с помощью различных наборов систем, включая навигацию и низкоуровневые контуры управления отдельными приводами робота. В частности, в рамках общей задачи управления шагающим роботом можно выделить две ключевые подзадачи: планирование движения на высоком уровне и планирование движения низкого уровня \cite{WalkingRobots}. В первой подзадаче стоит вопрос «куда двигаться», а вторая отвечает на вопрос «как двигаться» в направлении заданном в первой подзадаче \cite{bib1}.

Чтобы полностью использовать возможности шагающих механизмов, необходим регулятор, способный решать сложные задачи, связанные с динамической локомоцией, такие как управление телом во время коротких периодов стояния и управление движением ног на высокой скорости \cite{KIM2019}.

Этот первый регулятор находит оптимальные силы реакции в течение одного полного цикла движения походки, используя управление с прогнозирующими моделями со следующей простой моделью единичной массы:
\begin{align}
	& m \ddot{p} = \sum_{i=1}^{n_c}f_i-c_g,\\
	& \frac{\partial}{\partial t}(\mathbb{I} \omega) = \sum_{i=1}^{n_c} r_i \times f_i,
\end{align}
где $p, f_i$ и $c_g$ --- трёхмерные векторы, представляющие положение робота, силу реакции и гравитационное ускорение по отношению к глобальной системе координат. $m$ --- масса робота, $n_c$ --- количество контактов. $\mathbb{I} \in \mathbb{R}^{3\times 3}$ --- тензор вращательной инерции и $\omega$ --- угловая скорость движения тела. $r_i$ --- это положение $i$-й точки контакта относительно центра масс робота, которое эквивалентно плечу момента контактной силы.

Во втором процессе мы используем управления всем телом для достижения высокой пропускной способности управления за счёт использования динамики всего тела и высокочастотного управления с обратной связью. Эта более совершенная модель динамики будет определять более точные команды крутящего момента, чем модель единичной массы. Динамика нескольких тел может быть записана как:

\begin{equation}
	A \begin{bmatrix}
		\ddot{{q}}_f \\ \ddot{{q}}_j
	\end{bmatrix}
	+ b +g =
	\begin{bmatrix}
		0_6 \\ \tau
	\end{bmatrix} +
	J_c\T f_r,
\end{equation}
где $A, b, g, \tau, f_r $ и $J_c$ --- обобщённая матрица масс, сила Кориолиса, сила тяготения, совместный момент, увеличенная сила реакции и якобиан контакта, соответственно. $\ddot{{q}}_f \in \mathbb{R}^{6}$ --- ускорение плавающего тела и  $\ddot{{q}}_j \in \mathbb{R}^{n_j}$ --- вектор ускорений суставов, где $n_j$ - количество суставов.
\FloatBarrier
