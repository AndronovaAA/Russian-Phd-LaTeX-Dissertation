\chapter{Описание робота и модели}\label{ch:ch2}
\section{Шагающие роботы}\label{sec:ch2/sec1}
При рассмотрении системы четвероногих роботов как плавающей в пространстве системы с несколькими телами исследования в основном были сосредоточены на определении положения и ориентации основания, а также каждой ноги. Переменные состояния включают конфигурацию тела и углы сочленения ног. Управляющие силы в системе включают крутящие моменты в суставах и силы реакции на грунт.\cite{henze2016}, \cite{farshidian2017robust} Желаемая траектория движения тела и каждой ноги может быть сформирована с помощью предварительного планирования или ограничений. Помимо ограничений, связанных с контактом с землёй и трением,\cite{henze2017multi} задачи могут быть описаны в виде уравнений или уравнений неравенства ограничений, которые включают переменные состояния или крутящие моменты суставов. 

\section{Механика контакта, модель контакта}\label{sec:ch2/sect2}
\section{Уравнение Лагранжа и манипуляторные уравнения для шагающих роботов}\label{sec:ch2/sect3}

Уравнения Лагранжа имеют вид:

\begin{equation}
	\frac{d}{dt} \bigg( 
	\frac{\partial T }{\partial \dot{{q}}}
	\bigg) - 
	\frac{\partial T }{\partial {q}} = \tau
\end{equation}
%
где $T$ - кинетическая энергия, ${q}$ - вектор обобщённых координат, а $\tau$ - обобщённые крутящие моменты. Заметим, что кинетическая энергия может быть описана как $T = \frac{1}{2} \dot{{q}}\T {H} \dot{{q}}$, где ${H}$ - обобщённая матрица инерции. Матрица ${H}$ является симметричной, положительно-определённой и полного ранга.

Обобщённые силы $\tau$ могут быть сформированы декартовыми силами или декартовыми крутящими моментами. Мы можем описать соотношения между декартовой силой ${f}$ и связанной с ней обобщённой силой:

\begin{equation}
	\tau_i = \left( \frac{\partial {r}_i}{\partial {q}} \right)\T {f}_i,
\end{equation}
%
где ${r}_i = {r}_i({q})$ - вектор, описывающий положение точки приложения силы ${f}_i$, как функция обобщённых координат ${q}$.

Если мы определим якобиан ${J}_i = \frac{\partial {r}_i}{\partial {q}}$, мы можем переписать вышеприведённое соотношение как:
%
\begin{equation}
	\tau_i = ({J}^r_i)\T {f}_i.
\end{equation}
Мы можем описать соотношения между декартовым моментом ${m}$ и связанной с ним обобщённой силой:

\begin{equation}
	\tau_i = \left( \frac{\partial \omega_i}{\partial \dot{{q}}} \right)\T {m}_i,
\end{equation}
%
где $\omega_i = \omega_i({q}, \dot{{q}})$ - угловая скорость тела, к которому приложен декартов момент ${m}$.
Если мы определим Якобиан ${J}^\omega_i = \frac{\partial \omega_i}{\partial \dot{{q}}}$, мы можем переписать вышеприведённое соотношение как:
%
\begin{equation}
\tau_i = ({J}^\omega_i)\T {m}_i.
\end{equation}
Мы можем определить двигательный момент ${w}$:
\begin{equation}
	{w} = \begin{bmatrix}
		{f} \\\ {m}
	\end{bmatrix}.
\end{equation}
%
Мы можем описать соотношения между двигательным моментом ${w}$ и связанной с ним обобщённой силой:
%
\begin{equation}
	\tau_i = \begin{bmatrix}
		{J}^r_i \\\ {J}^\omega_i
	\end{bmatrix}\T
	{w}_i
	=
	{J}_i\T {w}_i.
\end{equation}
Заметим, что полная обобщённая сила может быть вычислена как:
%
\begin{equation}
	\tau = \sum {J}_i\T {w}_i.
\end{equation}
%
Уравнения Лагранжа с ограничениями имеют вид:
%
\begin{equation}
	\begin{cases}
		\frac{d}{dt} \bigg( 
		\frac{\partial T }{\partial \dot{{q}}}
		\bigg) - 
		\frac{\partial T }{\partial {q}} = \tau + \left( \frac{\partial {r}}{\partial {q}} \right)\T \lambda
		\\
		{r}({q}) = 0
	\end{cases},
\end{equation}
%
где ${r}({q}) = 0$ - ограничения, а $\lambda$ - силы реакции. Мы можем считать $\lambda$ конкатенацией всех сил реакции, связанных с ограничениями.

Уравнения манипулятора имеют вид:
%
\begin{equation}
	{H} \ddot{{q}} + {C} \dot{{q}} + {g} = \tau,
\end{equation}
%
где ${H}={H}({q})$ - обобщённая матрица инерции, ${C}\dot{{q}}$ - обобщённые инерционные силы, ${g}={g}({q})$ - обобщённые гравитационные силы.
Уравнения манипулятора имеют вид:
%
\begin{equation}
	\begin{cases}
		{H} \ddot{{q}} + {C} \dot{{q}} + {g} = \tau + \left( \frac{\partial {r}}{\partial {q}} \right)\T \lambda
		\\
		{r}({q}) = 0
	\end{cases}.
\end{equation}
%
Определив якобиан ограничений ${J} = \frac{\partial {r}}{\partial {q}}$, мы можем переписать уравнения:
%
\begin{equation}
	\begin{cases}
		{H} \ddot{{q}} + {C} \dot{{q}} + {g} = \tau + {J}\T \lambda
		\\
		{r}({q}) = 0
	\end{cases}.
\end{equation}
Дифференцируя ограничение ${r}({q})$, получаем:
%
\begin{equation}
	\frac{\partial {r}}{\partial {q}} \frac{\partial {q}}{\partial t} = 0
\end{equation}
%
Это можно записать в виде:
%
\begin{equation}
	{J} \dot{{q}} = 0
\end{equation}
%
Продифференцировав ещё раз, получим:
%
\begin{equation}
	{J} \ddot{{q}} + \dot{{J}} \dot{{q}} = 0.
\end{equation}
%
Мы можем заменить ограничения их вторыми производными:
%
\begin{equation}
	\begin{cases}
		{H} \ddot{{q}} + {C} \dot{{q}} + {g} = \tau + {J}\T \lambda
		\\
		{J} \dot{{q}} + \dot{{J}} \dot{{q}} = 0
	\end{cases}
\end{equation}
%
Это ДАУ в переменных ${q}$ и $\lambda$. Мы можем переписать его в векторно-матричной форме:
%
\begin{equation}
	\begin{bmatrix}
		{H} & -{J}\T \\\\
		{J} & 0
	\end{bmatrix}
	\begin{bmatrix}
		\ddot{{q}} \\
		\lambda
	\end{bmatrix}
	=
	\begin{bmatrix}
		\tau - {C} \dot{{q}} - {g} \\
		-\dot{{J}} \dot{{q}}
	\end{bmatrix}
\end{equation}
\section{Обратная задача динамики для шагающих роботов}\label{sec:ch2/sect4}
Обратная динамика для системы $\mathcal{G}$ - это задача нахождения управляющего входа ${u}$ при заданном состоянии $({q}, \dot{{q}})$ и желаемом значении производной высшего порядка $\ddot{{q}}$.

Динамика свободно плавающего робота с жёстким телом, подверженного внешним ограничениям, в общем случае задаётся следующим образом
\begin{equation}
	M \ddot{q} +h =S\T\tau + J_c\T \lambda,
\end{equation}
при $k$ ограничениях
\begin{equation}
	J_c \ddot{q} = b(q,\dot{q}),
\end{equation}
где $M \in \mathbb{R}^{n+6 \times n+6}$ - матрица инерции динамики жёсткого тела, $h \in \mathbb{R}^{n+6}$ - обобщённый вектор силы, содержащий эффекты Кориолиса, центробежные и гравитационные эффекты, $\tau \in \mathbb{R}^{n}$ - вектор воздействия и $S \in \mathbb{R}^{n \times n+6}$ - матрица выбора соединений, отражающая недостаточность действия - например, для большинства роботов с плавающей базой S будет матрицей тождества в первой подматрице $n \times n$, а в остальных местах - нули. 
$J_c \in \mathbb{R}^{k \times n+6}$ - Якобиан $k$ ограничений c $\lambda \in \mathbb{R}^{k}$ множителями Лагранжа, соответствующих ограничивающим силам.


\section{прямая задача динамики и математическое моделирования шагающих роботов}\label{sec:ch2/sect5}
(контакт как виртуальные пружины, контакт как связи, complementarity constraint и др.)
\section{управление шагающими роботами (по Киму)}\label{sec:ch2/sect6}
\FloatBarrier
