\chapter{Описание робота и модели}\label{ch:ch2}
\section{Шагающие роботы}\label{sec:ch2/sec1}
При рассмотрении системы четвероногих роботов как плавающей в пространстве системы с несколькими телами исследования в основном были сосредоточены на определении положения и ориентации основания, а также каждой ноги. Переменные состояния включают конфигурацию тела и углы сочленения ног. Управляющие силы в системе включают крутящие моменты в суставах и силы реакции на грунт.\cite{henze2016}, \cite{farshidian2017robust} Желаемая траектория движения тела и каждой ноги может быть сформирована с помощью предварительного планирования или ограничений. Помимо ограничений, связанных с контактом с землёй и трением,\cite{henze2017multi} задачи могут быть описаны в виде уравнений или уравнений неравенства ограничений, которые включают переменные состояния или крутящие моменты суставов. 

\section{Механика контакта, модель контакта}\label{sec:ch2/sect2}
\section{уравнение Лагранжа и манипуляторные уравнения для шагающих роботов}\label{sec:ch2/sect3}
\section{обратная задача динамики для шагающих роботов}\label{sec:ch2/sect4}
\section{прямая задача динамики и математическое моделирования шагающих роботов}\label{sec:ch2/sect5}
(контакт как виртуальные пружины, контакт как связи, complementarity constraint и др.)
\section{управление шагающими роботами (по Киму)}\label{sec:ch2/sect6}
\FloatBarrier
