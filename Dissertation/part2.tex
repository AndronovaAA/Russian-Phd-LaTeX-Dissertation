\chapter{Описание робота и модели}\label{ch:ch2}
\section{Шагающие роботы}\label{sec:ch2/sec1}


\section{Линеаризация}\label{sec:ch2/sect2}
Конфигурация шагающего робота может быть описана вектором обобщённых координат $q$. Обобщённые координаты и их производные $\dot q$ вместе образуют вектор состояния робота $x$. Взаимодействие робота с окружающей средой можно описать как ограничение на скорость изменения состояния \cite{mason2014full}: ${G} \dot x = 0$. Это мотивирует ортогональную декомпозицию вектора состояния:
%
\begin{equation}
	x = {N} z^* + {R} \zeta^* ,
\end{equation}
%
где $z^*$ - координаты состояния, изменяющиеся со временем, а $\zeta^* = \text{const}$; матрицы ${N}$ и ${R}$ - ортонормированные базисы в подпространстве нулей и подпространстве строк матрицы ограничений ${G}$.

Рассмотрим уравнение динамики $\dot z^* = f(z^*, \zeta^*)$ с параметрами $\zeta^* = \text{const}$ и его разложение Тейлора в окрестности точки $z^* = z_0$, $\zeta^* = \zeta_0$. Определим $z = z^* - z_0$ и $\zeta = \zeta^* - \zeta_0$;
Опустив квадратичные члены и члены высших порядков, но сохранив линейные и билинейные члены, запишем приближенную динамику в виде:
%
\begin{equation}
	\label{eq:model_linearization_1}
	\dot z = \frac{\partial f}{\partial z} z + \Delta {A}(\zeta) z + \frac{\partial f}{\partial \zeta} \zeta ,
\end{equation}
%
где $\Delta {A}(\zeta) = \frac{\partial}{\partial z} (\frac{\partial f}{\partial \zeta} \zeta)$. Мы оставляем билинейный член $\Delta {A}(\zeta) z$ потому, что $\zeta$ постоянна, а $\Delta {A}(\zeta)$ не обращается в нуль и влияет на устойчивость системы.

Модель \eqref{eq:model_linearization_1} включает в себя как мультипликативные, так и аддитивные неопределённости. Мы можем интерпретировать постоянные параметры $\zeta$ как неопределённые параметры модели, либо как статические компоненты состояния шагающего робота. В последнем случае мы приходим к модели шагающего робота на временном интервале между последовательными шагами.

Предположим, что основным источником неопределённости в рассматриваемой динамике является геометрия окружающей среды (опорной поверхности), с которой соприкасается робот. Пусть $K_s$ - точка на опорной поверхности, а $K_r$ - точка на теле робота, которая с ней соприкасается (т. е. точка на ноге робота). Пусть $e_K$ - вектор нормали к опорной поверхности в точке $K_s$, а $h_K(q) \in \mathbb{R}$ - вектор, проведённый из $K_r$ в $K_s$, спроецированный на $e_k$. Тогда мы можем моделировать неопределённость как:
%
\begin{equation}
	\label{eq:uncertain_contact}
	h_K(q) = \Delta h,
\end{equation}
%
где $\Delta h$ - неточность в положении точки $K_s$. Если робот соприкасается с опорной поверхностью, это означает, что конфигурация робота описывается не координатами $q$, а $q + \Delta q$, где $\Delta q$ - неточность конфигурации. Мы можем определить несоответствие состояния как 
$\Delta x = \begin{bmatrix}
	\Delta q \\ \Delta \dot q
\end{bmatrix}$
и находим, что связь между статическими состояниями и неточностью состояний имеет вид $\zeta = {R}\T \Delta x$. Написав разложение Тейлора для \eqref{eq:uncertain_contact} и его производной, находим:
%
\begin{align} 
	\label{eq:J_Delta_q} 
	{J}_K \Delta q = \Delta h,
	\\
	\label{eq:J_Delta_qdot} 
	{J}_K \Delta \dot q + \dot{{J}}_K \Delta q = \Delta \dot h,
\end{align}
%
где ${J}_K = \partial h_K / \partial q$.
Предположим, что $\Delta \dot h = 0$ и $\Delta h$ - неизвестная скалярная переменная с областью определения $-s_K \leq \Delta h \leq s_K$, где $s_K$ - положительный скаляр. Тогда мы можем вычислить расхождение состояний как решение по методу наименьших квадратов для \eqref{eq:J_Delta_q} и\eqref{eq:J_Delta_qdot}. Проецируя это решение на столбцы $R$, мы находим $\zeta$:
%
\begin{align}
	\zeta = {R}\T 
	\begin{bmatrix}
		{J}_K & 0 \\ \dot{{J}}_K & {J}_K
	\end{bmatrix}^+ 
	\begin{bmatrix}
		s_K \\ 0
	\end{bmatrix} \varsigma,
\end{align}
%
где $-1 \leq \varsigma \leq 1$. Учитывая определение $\Delta {A}(\zeta)$, мы можем написать:
\begin{equation}
	\Delta {A}(\zeta) = 
	\frac{\partial}{\partial z} 
	\left (
	\frac{\partial f}{\partial \zeta} {R}\T 
	\begin{bmatrix}
		{J}_K & 0 \\ \dot{{J}}_K & {J}_K
	\end{bmatrix}^+ 
	\begin{bmatrix}
		s_K \\ 0
	\end{bmatrix} \varsigma 
	\right )
	= \Tilde{A} \varsigma,
\end{equation}
%
где $\Tilde{A}$ - некоторая матрица, образующая базис в пространстве допустимых значений неопределённой матрицы $\Delta {A}$.

Наша цель - получить такие матрицы ${M}_1$, ${F}_1$ и ${N}_1$, где ${F}_1\T {F}_1 \leq {I}$, что для любого выбранного вектора $z$ множество всех векторов ${M}_1 {F}_1 {N}_1 z$ будет содержать множество $(\Tilde{A} \varsigma) z$. Учитывая сингулярное разложение $\Tilde{A} = {U}_A {\Sigma}_A {N}_1\T$, где 
${U}_A = \begin{bmatrix}
	{u}_1, ..., {u}_n
\end{bmatrix}$, 
${\Sigma}_A = \text{diag}(\sigma_1, ..., \sigma_n)$, мы определяем новую матрицу ${M}_1$:
%
\begin{equation}
	{M}_1 = 
	\begin{bmatrix}
		\sigma_1{u}_1, ..., \sigma_n{u}_n
	\end{bmatrix}.
\end{equation}
Таким образом, мы можем описать матрицу эквивалентным образом $\Delta {A} = {M}_1 {F}_1 {N}_1$, где ${F}_1\T{F}_1\leq {I}$.

\FloatBarrier
