\chapter{Математические эксперименты}\label{ch:ch4}
В этом разделе мы демонстрируем использование предложенных методов. Мы применим его для разработки робастного управления для четвероногого робота. 
\section{Описание робота}\label{sec:ch4/sect1}
Мы используем плоскую модель четвероногого робота (робот описывается в одном горизонтальном и одном вертикальном измерении), как показано на рисунке \ref{fig:FlatQuadruped}. Робот состоит из двух ног, прикреплённых к корпусу с помощью поворотного шарнира с приводом; каждая нога образована двумя жёсткими звеньями, соединёнными поворотным шарниром с приводом. Длина каждого звена $l_1 = 0,3 м$, а длина тела $l_2 = 0,5 м$. Масса тела робота составляет 10 кг, а масса каждого звена - 2 кг. Во время эксперимента ноги робота поддерживают контакт с землёй.

Состояние робота включает в себя положение и ориентацию его тела, углы наклона суставов и производные от них. Мы линеаризуем модель робота вокруг следующей конфигурации $q_1 =- \pi/6$, $q_2 = -\pi / 6$, $q_3 = \pi / 3$, и $q_4 = \pi / 3$.
\section{Описание эксперимента}\label{sec:ch4/sect2}
Сначала мы продемонстрируем работу предложенных методов. Мы решаем оптимизационную задачу \eqref{eq:thm1_OCP} для случая смешанной неопределённости (аддитивной и мультипликативной) и оптимизационную задачу \eqref{eq:thm3_OCP} только для мультипликативной неопределённости. Мы решаем обе задачи для различных значений $\epsilon_1$. На рисунке \ref{fig:cost} показано, как их соответствующие оптимальные затраты зависят от выбора $\epsilon_1$. Результаты показывают, что задача выполнима, а оптимальная стоимость как функция от $\epsilon_1$ описывает выпуклую кривую, что упрощает выбор оптимального значения этого параметра.

Во-вторых, мы показываем, как предложенные методы могут быть использованы для поиска наибольшего структурированного набора неопределённостей, которые может выдержать робастный линейный регулятор. Такая задача имеет ряд практических приложений. Она может быть использована для определения того, достигла ли конструкция регулятора своих пределов с точки зрения неопределённостей, которые он должен переносить. Задача также может быть решена для определения того, насколько велик набор неопределённостей, которые могут быть допустимы для данного конкретного робота и его конкретной конфигурации, что облегчает соответствующий анализ.
\section{Описание результатов эксперимента}\label{sec:ch4/sect3}