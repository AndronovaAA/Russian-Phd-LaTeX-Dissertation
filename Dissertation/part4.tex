\chapter{Обобщение метода для неопределённостей}\label{ch:ch4}
\section{Мультипликативная мягкая неопределённость}\label{sec:ch4/sect1}

В этом подразделе мы рассмотрим случай, когда неопределённости ограничены следующим образом:
%
\begin{equation}
	\label{eq:part2_nu_uncertainty}
	{F}_i\T{F}_i\leq \nu_i {I}, \ \ i=1,2,3;
\end{equation}
%
Мы можем предложить условия на изменение коэффициента регулятора и наблюдателя, которые гарантируют устойчивость при любом допустимом значении ${F}_1$ для заданного радиуса неопределённости $\nu$:
\begin{theorem}\label{thm:part2_LMI_2}
	Система (\ref{eq:part2_system}) асимптотически стабильна для всех матриц
	$\Delta {A}_n={M}_1{F}_1{N}_1$, 
	$\Delta {B}= {M}_2{F}_2{N}_2$, 
	$\Delta {C} = {M}_3{F}_3{N}_3$
	где
	${F}_i\T{F}_i\leq \nu_i {I}$ для $i=1,2,3$
	если существуют такие позитивно-определённые матрицы ${Q}_1$, ${P}_2$и позитивные скаляры
	$\gamma_1>0$, $\gamma_2>0$, $\gamma_3>0$, $\bar{\mu}_1>0$, $\bar{\mu}_2>0$, $\bar{\mu}_3>0$, и $\epsilon_1 > 0$ такие, чтобы следующие линейные матричные неравенства выполнялись:
	%
	\begin{equation}
		\label{eq:thm4_final_LMI}
		\begin{bmatrix}
			{\Lambda}_1 & 0 & {M}_1 & {M}_2&0 & {Q}_1{N}_1\T & {\Lambda}_3 &{\Lambda}_4 & {B}\hat{{K}}_z & 0\\
			* & {\Lambda}_2 & {P}_2{M}_1 & {P}_2{M}_2 & \hat{{L}}_z{M}_3& 0& 0&0&0&{I} \\
			* & * & -\gamma_1{I} & 0&0&0&0&0&0&0\\
			* & * &*  & -\gamma_2{I}&0&0&0&0&0&0\\
			*& * & * &*  & -\gamma_3{I}&0&0&0&0&0\\
			* &* & * & * & *&-\bar{\mu}_1{I}&0&0&0&0\\
			* & * & * &*& *&*&-\bar{\mu}_2{I}& 0&{N}_2\hat{{K}}_z&0\\
			*&*&* &* & * & * & *&-\bar{\mu}_3{I}&0&0\\
			* & * & *&*&* & *&*&*&-\frac{1}{\epsilon_1}{Q}_1&0\\
			* & * & * & *&*&*&*&*&*&-\epsilon_1{Q}_1\\
		\end{bmatrix}<0,
	\end{equation}
	%
	где
	%
	\begin{align}
		{\Lambda}_3&=\hat{{K}}_z\T{N}_2\T,\\ {\Lambda}_4&={Q}_1{N}\T{N}_3\T,
	\end{align}
	%
	и ${\Lambda}_1$, ${\Lambda}_2$ определены в \eqref{eq:Lambda_1} и \eqref{eq:Lambda_2}. 
\end{theorem}
\begin{proof}
	Доказательство этой теоремы мы начинаем аналогично доказательству теоремы \ref{thm:part2_LMI_1}; шаги, представленные уравнениями \eqref{eq:thm3_Lyapunov_candidat}-\eqref{eq:eta_3_thm3}, те же самые. Отсюда, используя \eqref{eq:part2_nu_uncertainty}, мы получаем:
	\begin{align}
		\label{eq:eta1_bound}
		\eta_1\T\eta_1 &\leq \chi\T \begin{bmatrix}
			{N}_1\T \\ 0  
		\end{bmatrix}\nu_1\begin{bmatrix}
			{N}_1 \ \ 0  
		\end{bmatrix} \chi,
		\\
		\eta_2\T\eta_2 &\leq \chi\T \begin{bmatrix}
			{K}_z\T{N}_2 \\ -{K}_z\T{N}_2\T  
		\end{bmatrix}\nu_2\begin{bmatrix}
			{N}_2{K}_z & -{N}_2{K}_z
		\end{bmatrix} \chi,
		\\
		\eta_3\T\eta_3 &\leq \chi\T \begin{bmatrix}
			-{N}\T{N}_3\T \\ 0  
		\end{bmatrix}\nu_3\begin{bmatrix}
			-{N}{N}_3 \ \ 0  
		\end{bmatrix} \chi.
	\end{align}
	%
	Используя S-процедуру и определяя $\bar{\mu}_i=\frac{1}{\mu_i}$, напишем:
	\begin{multline}
		\begin{bmatrix}
			{\Theta}_1 & -{P}_1{B}{K}_z & {P}_1{M}_1 & {P}_1{M}_2 &0 \\
			* &    {\Theta}_2 & {P}_2{M}_1 & {P}_2{M}_2 & {P}_2{L}_z{M}_3\\
			* & * & 0 & 0&0\\
			* & * & * & 0&0 \\
			* & * & * & *&0
		\end{bmatrix} + \\
		\begin{bmatrix}
			\mathcal{X}_{\mu}& -\mu_2{K}_z\T{N}_2\T{N}_2{K}_z &0 &0 & 0\\
			*&\mu_2{K}_z\T{N}_2\T{N}_2{K}_z&0&0&0\\
			*&*&-\gamma_1{I}&0&0\\
			*&*&*&-\gamma_2{I}&0\\
			*&*&*&*&-\gamma_3{I}\\
		\end{bmatrix} 
		<0,
	\end{multline}
	где
	%
	\begin{equation}
		\mathcal{X}_{\mu}=\mu_1{N}_1\T{N}_1 +\mu_2{K}_z\T{N}_2\T{N}_2{K}_z+\mu_3{N}\T{N}_3\T{N}_3{N},
	\end{equation}
	и $\mu_i=\gamma_i\nu_i$ для $i=1,2,3$, то используя дополнение Шура
	%
	\begin{equation}
		\label{eq:thm4_after_Schur}
		\begin{bmatrix}
			{\Theta}_1 & -{P}_1{B}{K}_z & {P}_1{M}_1 & {P}_1{M}_2 & 0 & {N}_1\T & {K}_z\T{N}_2\T & {N}\T{N}_3\T 
			\\
			* & {\Theta}_2 & {P}_2{M}_1 & {P}_2{M}_2 & {P}_2{L}_z{M}_3&0& -{K}_z\T{N}_2\T& 0\\
			* & * & -\gamma_1{I} & 0&0&0&0&0\\
			* & * & * & -\gamma_2{I}&0&0&0&0\\
			* & * & * & *&-\gamma_3{I}&0&0&0\\
			* & * & * & *&*&-\bar{\mu}_1{I}&0&0\\
			* & * & * & *&*&*&-\bar{\mu}_2{I}&0\\
			*&* & * & * & *&*&*&-\bar{\mu}_3{I}\\
		\end{bmatrix}<0.
	\end{equation}
	%
	Продолжая, как и в предыдущих доказательствах, использовать отношение Юнга, дополнение Шура и замену переменных, мы приходим к окончательному линейному матричному неравенству \eqref{eq:thm4_final_LMI}.
\end{proof}
В отличие от предыдущих задач оптимизации, здесь не существует естественной формулировки целевой функции. Мы предлагаем два варианта целевой функции:
\begin{align}
	\label{eq:cost_lin}
	J_\text{lin} &= \sum_{i=1}^{3}\left(\bar{\mu}_i+\gamma_i\right) \\ 
	\label{eq:cost_quad}
	J_\text{quad} & =  \sum_{i=1}^{3}\left((\bar{\mu}_i-\varpi)^2+(\gamma_i-\varpi)^2\right) + \varpi^2,
\end{align}
%
где $\varpi$ - свободная переменная. Целью обеих функций затрат является максимизация $\nu_i$, которая достигается минимизацией либо $J_\text{lin}$, либо $J_\text{quad}$. Максимизируя $\nu_i$, мы достигаем устойчивости к большему набору неопределённых матриц $\Delta {A}_n$, $\Delta {B}$ и $\Delta {C}$.

Кроме того, может быть интересно ограничить набор неопределённых матриц, ограничив $\nu_i \leq 1$. Утверждение $\nu_i=\frac{1}{\bar{\mu_i}\gamma_i} \leq 1$ преобразуется в $\frac{1}{\bar{\mu}_i}\leq \gamma_i$, и, используя дополнение Шура, мы получаем линейное ограничение: 
%
\begin{equation}
	\label{eq:mu_gamma_limit}
	\begin{bmatrix}
		-\gamma_i & 1 \\
		1 & -\bar{\mu}_i
	\end{bmatrix}
	\leq 0. \end{equation}
%
Выбирая линейную целевую функцию и ограничивая $\nu_i$, задача оптимизации приобретает вид:
%
\begin{equation}
	\label{eq:thm4_OCP}
	\begin{aligned}
		& \underset{\bar{\mu}_i,\gamma_i,{Q}_1, {P}_2,\hat{{K}} , \hat{{L}} }{\text{minimize}}
		& &  \sum_{i=1}^{3}\left(\bar{\mu}_i+\gamma_i\right), \\
		& \text{subject to}
		& & \begin{cases}
			{Q}_1>0, \ \
			{P}_2>0, \ \
			\bar{\mu}_i>0, \ \
			\gamma_i>0, \ \
			\text{для} \ \ i=1,2,3; \\
			\text{условия \eqref{eq:thm4_final_LMI}, \eqref{eq:mu_gamma_limit} }.
		\end{cases}
	\end{aligned}
\end{equation}
%
Данная задача оптимизации имеет параметр $\epsilon_1$ который мы можем найти решетчатым поиском как в \ref{rm:griding_search}.
\begin{remark}
	Эту задачу можно рассматривать как обобщение и ослабление исходной задачи, рассмотренной в предыдущем подразделе. Вместо того чтобы искать управление, устойчивое к заданной неопределённости, мы пытаемся найти управление, устойчивое к наибольшей возможной неопределённости.
\end{remark}


В этом разделе мы демонстрируем использование предложенных методов. Мы применим его для разработки робастного управления для четвероногого робота. 

\section{Аддитивная неопределённость}\label{sec:ch4/sect2}
Рассмотрим следующую систему:
%
\begin{equation}
	\label{eq:part1_linear_dynamics}
	\begin{cases}
		\dot z=({A}_n+\Delta {A}_n)z + {A}_r\zeta + {B}u,\\
		y={C}{N}z+{C}{R}\zeta,
	\end{cases}
\end{equation}
%
где $z$ это динамические состояния, $\zeta = \text{const}$ - статические состояния и $\Delta {A}_n$ представляет мультипликативную модель неопределённостей и имеет следующую структуру:
%
\begin{equation}
	\label{eq:part1_uncertainty}
	\Delta {A}_n={M}_1{F}_1{N}_1 \quad \text{и} \quad {F}_1\T{F}_1\leq \nu {I},
\end{equation}
%
где ${M_1} \in \mathbb{R}^{n_z \times d}$ и 
${N_1} \in \mathbb{R}^{d \times n_z}$ известные матрицы, ${F}_1$ - неизвестная матрица ограниченная по норме и $\nu$ - неизвестный радиус - скаляр, определяющий лимит наложенный на норму ${F}_1$.

Следуя \cite{SAVIN2021} мы можем ввести наблюдатель Люнберга:
%
\begin{equation}
	\begin{bmatrix}
		\dot{\hat{z}} \\
		\dot{\hat{\zeta}}
	\end{bmatrix}=\begin{bmatrix}
		{A}_n & {A}_r \\
		0 & 0
	\end{bmatrix}
	\begin{bmatrix}
		\hat{z}\\ \hat{\zeta}
	\end{bmatrix}
	+  \begin{bmatrix}
		{B}\\0
	\end{bmatrix}u + {L} \left( y-\begin{bmatrix}
		{C}{N} & {C}{R}
	\end{bmatrix} \begin{bmatrix}
		\hat{z}\\ \hat{\zeta}
	\end{bmatrix} \right),
\end{equation}
%
где ${L}$ коэффициент наблюдателя.

Определим ошибку оценки состояния как $e = [ (z-\hat{z})\T \ \ (\zeta-\hat{\zeta})\T ]\T$ и введем блочную матрицу:
${S} = \begin{bmatrix}
	{I} \\ 0
\end{bmatrix}$, 
${E}=[ {N} \ \ {R}]$, и 
$
{A}_c=    \begin{bmatrix}
	{A}_r  & {A}_{\rho} \\
	0  & 0
\end{bmatrix}
$
записываем ошибку наблюдателя динамики:
%
\begin{equation}
	\label{eq:part1_error_dynamics}
	\dot e= ({A}_e-{L}{C}{E})e +{S}\Delta {A}_n z.
\end{equation}
%
Вводим закон управления:
%
\begin{equation}
	u={K}_z \hat{z}+{K}_{\zeta} \hat{\zeta},
\end{equation}
%
и определяем ${K}=\begin{bmatrix}
	{K}_z & {K}_{\zeta}
\end{bmatrix}$ записываем динамическую систему с обратной связью:
%
\begin{equation}
	\label{eq:part1_active_dynamics}
	\dot{z}=({A}_n+\Delta {A}_n +{B}{K}_z)z-{B}{K}e+({A}_r+{B}{K}_{\zeta})\zeta.
\end{equation}
%
Коэффициент регулятора ${K}_{\zeta}$ может быть выбран как:
%
\begin{equation}
	\label{eq:part1_static_control}
	{K}_{\zeta}=-{B}^{\dagger}{A}_r.
\end{equation}
%
Пока столбцы${A}_r$ лежат в подпространстве столбцов ${B}$ и существует точная оценка состояния, данный закон управления сводит на нет эффект $\zeta$ на динамику. В данном случае,отмечая что ${K}_z={K}{S}$,  мы можем представить ошибку наблюдателя и динамику робота как систему уравнений:
%
\begin{equation}
	\label{eq:part1_system}
	\begin{bmatrix}
		\dot{z} \\ \dot{e}
	\end{bmatrix}=\begin{bmatrix}
		({A}_n+\Delta {A}_n +{B}{K}{S}) & {B}{K} \\
		{S} \Delta {A}_n & ({A}_e-{L}{C}{E})        \end{bmatrix}\begin{bmatrix}
		z \\ e
	\end{bmatrix}.
\end{equation}
%
Рассмотрим проблему нахождения таких коэффициентов регулятора и наблюдателя, которые будут давать устойчивую систему для всех допустимых значений $\Delta {A}_n$.

\subsection{Единичная неопределённость}\label{sec:ch4/sect2/sub1}

Начнём с рассмотрения случая, когда неопределённость строго ограничена неравенством ${F}_1\T{F}_1\leq {I}$, которое мы называем единичной неопределённостью. В этом случае следующая теорема даёт нам достаточное условие устойчивости, которое может быть непосредственно использовано при проектировании регуляторов и коэффициентов усиления наблюдателей и представлено в виде линейного матричного неравенства с параметром.
\begin{theorem}\label{thm:part1_LMI_1}
	Система \eqref{eq:part1_system} асимптотически устойчива для всех матриц $\Delta {A}_n={M}_1{F}_1{N}_1$ с ${F}_1\T{F}_1\leq {I}$, если существуют положительно-определённые матрицы ${Q}_1>0$, ${P}_2>0$, матрицы $\hat{{K}}, \hat{{L}}$ и скаляры $\epsilon_1>0,\epsilon_2>0,\epsilon_3>0$ такие, что следующее линейное матричное неравенство выполнимо: 
	%
	\begin{equation}
		\label{eq:thm1_main_LMI}
		\begin{bmatrix}    
			{\Psi}_1  & 0 & {\Xi} & 0 &  {Q}_1{N}_1\T & {Q}_1{N}_1\T & 0\\
			* & {\Psi}_2 & 0 & {I} & 0 & 0 & {P}_2{S}{M}_1\\
			* & * &  -\frac{1}{\epsilon_1}{H} & 0 & 0 &0 & 0\\
			* & * & * & -\epsilon_1{H} & 0 & 0 & 0 \\
			* & * & * & * & -\epsilon_2 {I} & 0 & 0 \\       * & * & * & * & *& -\epsilon_3 {I} & 0 \\
			* & * & * & * & *& * & -\frac{1}{\epsilon_3} {I}
		\end{bmatrix} <0,
	\end{equation}
	%
	где
	%
	\begin{equation}
		\label{eq:H_Xi_variables}
		{H} = \begin{bmatrix}
			{Q}_1 & 0 \\
			0 & {I}
		\end{bmatrix}, \ \ 
		{\Xi} = \begin{bmatrix}
			{B}\hat{{K}} & {B}{K}_{\zeta} \end{bmatrix},
	\end{equation}
	%
	\begin{align}
		\label{eq:Psi_1}
		{\Psi}_1&={Q}_1{A}_n\T+{A}_n{Q}_1+{B}\hat{{K}}+\hat{{K}}\T{B}\T  +\epsilon_2{M}_1{M}_1\T, \\
		\label{eq:Psi_2}
		{\Psi}_2 &={A}_e\T{P}_2+{P}_2{A}_e-\hat{{L}}{CE}-{E}\T{C}\T\hat{{L}}\T,
	\end{align}
	%
	и коэффициенты регулятора и наблюдателя находятся как ${L}={P}^{-1}_2\hat{{L}}$
	и ${KS}=\hat{{K}}{Q}^{-1}_1$.  
\end{theorem}
Используя теорему \ref{thm:part1_LMI_1} и добавляя выпуклую целевую функцию, мы формулируем робастную конструкцию управления в виде полуопределённой программы:
%
\begin{equation}
	\label{eq:thm1_OCP}
	\begin{aligned}
		& \underset{\epsilon_2,\alpha,\beta, {Q}_1, {P}_2,\hat{{K}} , \hat{{L}} }{\text{minimize}}
		& & \operatorname{tr}({Q}_1\T{W}_z{Q}_1)+ \operatorname{tr}({P}_2\T{W}_e{P}_2)+ \operatorname{tr}(\hat{{K}}\T{W}_k\hat{{K}})+\operatorname{tr}(\hat{{L}}\T{W}_l\hat{{L}}), \\
		& \text{subject to}
		& & \begin{cases}
			{Q}_1>0, \ \
			{P}_2>0, \ \
			\epsilon_2>0, \ \
			\alpha>0, \ \
			\beta>0, \\
			\text{условие \eqref{eq:thm1_final_LMI_ab} },
		\end{cases}
	\end{aligned}
\end{equation}
где ${W}_z,{W}_e,{W}_k$ и ${W}_l$ - весовые матрицы. 
\subsection{Мягкая неопределённость}\label{sec:ch4/sect2/sub2}
Пусть матрица неопределённости $\Delta {A}_n$ определяется как:
%
\begin{equation}
	\label{eq:nu_uncertainty}
	\Delta {A}_n={M}_1{F}_1{N}_1 \quad \text{и} \quad {F}_1\T{F}_1\leq \nu {I}.
\end{equation}
%
Мы можем предложить условия на усиление регулятора и наблюдателя, которые гарантируют устойчивость при любом допустимом значении ${F}_1$ для заданного радиуса неопределённости $\nu$:
%
\begin{theorem}\label{thm:part1_LMI_2}
	Система \eqref{eq:part1_system}
	асимптотически устойчива для любой $\Delta {A}_n =$${M}_1{F}_1{N}_1$ с ${F}_1\T{F}_1\leq \nu {I}$, если существуют положительно-определённые матрицы ${Q}_1>0$, ${P}_2>0$, матрицы $\hat{{K}}, \hat{{L}}$ и скаляр $\bar{\epsilon}>0$ такие, что следующее линейное матричное неравенство выполнимо: 
	%
	\begin{equation}
		\label{eq:thm2_final_LMI}
		\begin{bmatrix}    
			{\Upsilon}_1  & 0 & {\Xi} & 0 &  {Q}_1{N}_1\T & {M}_1 & 0\\
			* & {\Psi}_2 & 0 & {I} & 0 & 0 & {P}_2{S}{M}_1\\
			* & * &  -\frac{1}{\epsilon_1}{H} & 0 & 0 &0 & 0\\
			* & * & * & -\epsilon_1{H} & 0 & 0 & 0 \\
			* & * & * & * & -\frac{\bar{\epsilon}}{2}{I} & 0 & 0 \\       * & * & * & * & *&  -\bar{\epsilon}{I} & 0 \\
			* & * & * & * & *& * &  -\bar{\epsilon}{I}
		\end{bmatrix} <0,
	\end{equation}
	%
	где
	%
	\begin{equation}
		\label{eq:Upsilon_1}
		{\Upsilon}_1={Q}_1{A}_n\T+{A}_n{Q}_1+{B}\hat{{K}}+\hat{{K}}\T{B}\T, 
	\end{equation}
	Матрицы ${\Psi}_2$, ${\Xi}$ те же, что и в уравнениях \eqref{eq:Psi_2},\eqref{eq:H_Xi_variables},
	а коэффициенты регулятора и наблюдателя задаются ${L}={P}^{-1}_2\hat{{L}}$.
	и ${KS}=\hat{{K}}{Q}^{-1}_1$.
\end{theorem}
В качестве целевой функции мы можем использовать минимизацию $\bar{\epsilon}$, что эквивалентно максимизации $\nu$, так как $\nu=\frac{1}{\bar{\epsilon}^2}$. Тогда задача оптимизации приобретает вид:
%
\begin{equation}
	\label{eq:thm2_OCP}
	\begin{aligned}
		& \underset{\bar{\epsilon},{Q}_1, {P}_2,\hat{{K}} , \hat{{L}} }{\text{minimize}}
		& &  \bar{\epsilon}, \\
		& \text{subject to}
		& & \begin{cases}
			{Q}_1>0, \ \
			{P}_2>0, \ \
			\bar{\epsilon}>0, \\
			\text{condition \eqref{eq:thm2_final_LMI} }.
		\end{cases}
	\end{aligned}
\end{equation}
\section{Эксперименты}\label{sec:ch4/sect3}
Сначала мы продемонстрируем работу предложенных методов. Мы решаем оптимизационную задачу \eqref{eq:thm1_OCP} для случая смешанной неопределённости (аддитивной и мультипликативной) и оптимизационную задачу \eqref{eq:thm3_OCP} только для мультипликативной неопределённости. Мы решаем обе задачи для различных значений $\epsilon_1$. На рисунке \ref{fig:cost} показано, как их соответствующие оптимальные затраты зависят от выбора $\epsilon_1$. Результаты показывают, что задача выполнима, а оптимальная стоимость как функция от $\epsilon_1$ описывает выпуклую кривую, что упрощает выбор оптимального значения этого параметра.

Во-вторых, мы показываем, как предложенные методы могут быть использованы для поиска наибольшего структурированного набора неопределённостей, которые может выдержать робастный линейный регулятор. Такая задача имеет ряд практических приложений. Она может быть использована для определения того, достигла ли конструкция регулятора своих пределов с точки зрения неопределённостей, которые он должен переносить. Задача также может быть решена для определения того, насколько велик набор неопределённостей, которые могут быть допустимы для данного конкретного робота и его конкретной конфигурации, что облегчает соответствующий анализ.

\subsection{Параметры эксперимента}\label{sec:ch4/sect3/sub1}
\subsection{Результаты эксперимента}\label{sec:ch4/sect3/sub2}
