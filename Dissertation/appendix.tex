\chapter{Доказательство}\label{app:A}
\begin{comment}
Выражение \eqref{eq:Young_conjugated} может быть записано как:
%
\begin{multline}
	\begin{bmatrix}
		{\Pi}_1 & 0 \\ 0 & {\Sigma}_2 
	\end{bmatrix}+
	\begin{bmatrix}
		0&-{BK}\\-({BK})\T&0
	\end{bmatrix}+
	\begin{bmatrix}
		{M}_1{F}_1{N}_1{Q}_1 + {Q}_1({M}_1{F}_1{N}_1)\T&0 \\ 0& 0
	\end{bmatrix}
	\\ + \begin{bmatrix}
		0& {Q}_1({S} {M}_1{F}_1{N}_1)\T{P}_2 \\
		{P}_2{M}_1{F}_1{N}_1{Q}_1 - ({BK})\T & 0
	\end{bmatrix}<0,
\end{multline}
%
раскрываем:
%
\begin{multline}
	\label{eq:Young_expand}
	\begin{bmatrix}
		{\Pi}_1 & 0 \\ 0 & {\Sigma}_2 
	\end{bmatrix}+ \begin{bmatrix}
		-{BK} \\ 0
	\end{bmatrix}\begin{bmatrix}
		0 \\ {I}
	\end{bmatrix}\T
	+\begin{bmatrix}
		0 \\ {I} 
	\end{bmatrix}\begin{bmatrix}
		-{BK} \\0
	\end{bmatrix}\T+\begin{bmatrix}
		{Q}_1{N}_1\T \\0
	\end{bmatrix}{F}_1\T\begin{bmatrix}
		{M}_1 \\0
	\end{bmatrix}\T \\ + \begin{bmatrix}
		{M}_1 \\0
	\end{bmatrix}{F}_1\begin{bmatrix}
		{Q}_1{N}_1\T \\0
	\end{bmatrix}\T+\begin{bmatrix}
		{Q}_1{N}_1\T \\ 0
	\end{bmatrix}{F}_1\T\begin{bmatrix}
		0 \\ {P}_2{S}{M}_1
	\end{bmatrix}\T+  \begin{bmatrix}
		0\\ {P}_2{S}{M}_1
	\end{bmatrix}{F}_1\begin{bmatrix}
		{Q}_1{N}_1\T \\0
	\end{bmatrix}\T<0.
\end{multline}
%
Используя \eqref{eq:Young_relation_BMI} из леммы \ref{lemma:Young} \cite{KHELOUFI2013}, находим:
%
\begin{multline}
	\label{eq:thm1_term_1}
	\begin{bmatrix}
		-{BK} \\ 0
	\end{bmatrix}\begin{bmatrix}
		0 \\ {I}
	\end{bmatrix}\T
	+\begin{bmatrix}
		0 \\ {I} 
	\end{bmatrix}\begin{bmatrix}
		-{BK} \\0
	\end{bmatrix}\T 
	\\ 
	\leq
	\begin{bmatrix}
		-{BK} \\ 0
	\end{bmatrix} \epsilon_1 {H} \begin{bmatrix}
		-{BK} \\ 0
	\end{bmatrix}\T +\begin{bmatrix}
		0 \\ {I}
	\end{bmatrix}\frac{1}{\epsilon_1}{H}^{-1}\begin{bmatrix}
		0 \\ {I}
	\end{bmatrix}\T =
	\begin{bmatrix}
		{BK}\epsilon_1{H}{K}\T{B}\T & 0 \\
		0 & \frac{1}{\epsilon_1}{H}^{-1}
	\end{bmatrix}.
\end{multline}
%
Из неравенства \eqref{eq:Young_robust_2} и леммы \ref{lemma:Young} с $\nu=1$ получаем:
%
\begin{multline}
	\label{eq:thm1_term_2}
	\begin{bmatrix}
		{Q}_1{N}_1\T \\0
	\end{bmatrix}{F}_1\T\begin{bmatrix}
		{M}_1 \\0
	\end{bmatrix}\T+\begin{bmatrix}
		{M}_1 \\0
	\end{bmatrix}{F}_1\begin{bmatrix}
		{Q}_1{N}_1\T \\0
	\end{bmatrix}\T  \\ \leq \frac{1}{\epsilon_2}\begin{bmatrix}
		{Q}_1{N}_1\T \\0
	\end{bmatrix}\begin{bmatrix}
		{Q}_1{N}_1\T \\ 0
	\end{bmatrix}\T +\epsilon_2 \begin{bmatrix}
		{M}_1 \\0
	\end{bmatrix}\begin{bmatrix}
		{M}_1 \\0
	\end{bmatrix}\T.
\end{multline}
%
Похожим образом:
%
\begin{multline}
	\label{eq:thm1_term_3}
	\begin{bmatrix}
		{Q}_1{N}_1\T \\0
	\end{bmatrix}{F}_1\T\begin{bmatrix}
		0 \\ {P}_2{S}{M}_1
	\end{bmatrix}\T+\begin{bmatrix}
		0\\ {P}_2{S}{M}_1
	\end{bmatrix}{F}_1\begin{bmatrix}
		{Q}_1{N}_1\T \\0
	\end{bmatrix}\T \\ \leq  \frac{1}{\epsilon_3}\begin{bmatrix}
		{Q}_1{N}_1\T \\0
	\end{bmatrix}\begin{bmatrix}
		{Q}_1{N}_1\T \\ 0
	\end{bmatrix}\T +\epsilon_3 \begin{bmatrix}
		0 \\ {P}_2{S}{M}_1
	\end{bmatrix}\begin{bmatrix}
		0 \\ {P}_2{S}{M}_1
	\end{bmatrix}\T.
\end{multline}
%
Подставляя \eqref{eq:thm1_term_1},\eqref{eq:thm1_term_2} и \eqref{eq:thm1_term_3} в \eqref{eq:Young_expand} получаем выражение \eqref{eq:thm1_LMI_after_Young}.

\clearpage
\refstepcounter{chapter}

\end{comment}
CVXPY - это встроенный в Python язык моделирования выпуклых оптимизационных задач с открытым исходным кодом. Он позволяет выразить задачу естественным образом, следуя математике, а не в ограничительной стандартной форме, требуемой решателями.