\chapter{Аддитивная неопределённость c динамической обратной связью}\label{ch:ch5}
\section{Динамическая обратная связь}\label{sec:ch5/sect1}
Рассмотрим следующую систему:
\begin{equation}
	\label{eq:part5_linear_dynamics}
	\begin{cases}
		\dot z={A}_N {z} + {A}_R {\zeta} + {B} {u},\\
		y = {C} {N} {z} + {C} {R} {\zeta},
	\end{cases}
\end{equation}

Введём регулятор с динамической обратной связью:
\begin{equation}
	\label{eq:part5_controller}
	\begin{cases}
		\dot{{z}}_K = {A}_K {z}_K + {B}_K {y},\\
		{u} = {C}_K {z}_K + {D}_K {y},
	\end{cases}
\end{equation}
Мы можем переписать предыдущую систему уравнений как:
\begin{align}
	&\begin{bmatrix}
		{\dot{z}} \\ {\dot{z}}_K
	\end{bmatrix}
	=
	\begin{bmatrix}
		{A}_N & 0 \\
		0 & {A}_K
	\end{bmatrix}
	\begin{bmatrix}
		{z} \\ {z}_K 
	\end{bmatrix}
	+
	\begin{bmatrix}
		{B} & 0 \\
		0 & {B}_K 
	\end{bmatrix}
	\begin{bmatrix}
		{u} \\ {y}
	\end{bmatrix}
	+
	\begin{bmatrix}
		{A}_R \\ 0 
	\end{bmatrix}
	\zeta
	\\
	& \begin{bmatrix}
		{I} & -{D}_K \\
		0 & {I}
	\end{bmatrix}
	\begin{bmatrix}
		{u} \\ {y}
	\end{bmatrix}
	=
	\begin{bmatrix}
		0 & {C}_K \\
		{C} {N} & 0
	\end{bmatrix}
	\begin{bmatrix}
		{z} \\ {z}_K
	\end{bmatrix}
	+
	\begin{bmatrix}
		0 \\ {C} {R}
	\end{bmatrix}{\zeta}
\end{align}

где матрица $\bigl[ \begin{smallmatrix}  {I} & -{D}_K \\ 0 & {I} \end{smallmatrix} \bigr]$ должна быть невырожденной, какой она и является:
\begin{equation}
	\begin{gathered}
		\begin{bmatrix}
			{I} & -{D}_K \\ 0 & {I}
		\end{bmatrix}^{-1} = 
		\\
		\begin{bmatrix}
			{I} + {I}(-{D}_K)({I}-0*{I}(-{D}_K))*0*{I} & -{I}(-{D}_K)({I}-0*{I}(-{D}_K))^{-1}\\
			-({I} - 0*{I}({D}_K))^{-1}*0*{I} &
			{I}-(0*{I}(-{D}_K))^{-1}
		\end{bmatrix}= 
		\begin{bmatrix}
			{I} & {D}_K \\ 0 & {I}
		\end{bmatrix}
	\end{gathered}
\end{equation}
Подставим (6) в (5):
\begin{align}
	& \begin{bmatrix}
		{\dot{z}} \\ {\dot{z}}_K 
	\end{bmatrix}
	=\left(
	\begin{bmatrix}
		{A}_N & 0 \\
		0 & {A}_K
	\end{bmatrix}
	+
	\begin{bmatrix}
		{B} & 0 \\
		0 & {B}_K
	\end{bmatrix}
	\begin{bmatrix}
		{I} & {D}_K \\
		0 & {I}
	\end{bmatrix}
	\begin{bmatrix}
		0 & {C}_K \\
		{C}{N} & 0
	\end{bmatrix}
	\right)
	\begin{bmatrix}
		{z} \\ {z}_K
	\end{bmatrix}
	+ \\
	& \left(
	\begin{bmatrix}
		{B} & 0 \\
		0 & {B}_K
	\end{bmatrix}
	\begin{bmatrix}
		{I} & {D}_K \\
		0 & {I}
	\end{bmatrix}
	\begin{bmatrix}
		0 \\ {C}{R}
	\end{bmatrix}
	+
	\begin{bmatrix}
		{A}_R \\ 0 
	\end{bmatrix}\right)
	{\zeta}
\end{align}
\begin{align}
	&\begin{bmatrix}
		{\dot{z}} \\ {\dot{z}}_K 
	\end{bmatrix}
	=
	\begin{bmatrix}
		{A}_N + {B}{D}_K{C}{N} & {B}{C}_K \\
		{B}_K{C}{N} &{A}_K
	\end{bmatrix}
	\begin{bmatrix}
		{z} \\ {z}_K 
	\end{bmatrix}
	+
	\begin{bmatrix}
		{A}_R + {B}{D}_K{C}{R}\\ {B}_K{C}_R
	\end{bmatrix}
	{\zeta}
\end{align}
Обозначим:
\begin{align}
	{A}_{cl} = \begin{bmatrix}
		{A}_N + {B}{D}_K{C}{N} & {B}{C}_K \\
		{B}_K{C}{N} & {A}_K
	\end{bmatrix}
	\\
	{z}_{cl} = \begin{bmatrix}
		{\dot{z}} \\ {\dot{z}}_K 
	\end{bmatrix}
	\\
	{F}_{cl} = \begin{bmatrix}
		{A}_R + {B}{D}_K{C}{R}\\ {B}_K{C}_R
	\end{bmatrix}
\end{align}
Мы можем переписать (8), как:
\begin{align}
	&{\dot{z}}_{cl} = {A}_{cl} {z}_{cl}+ {F}_{cl}\zeta
\end{align}
\begin{theorem}
	Система стабильна
\end{theorem}
\begin{proof}
	Введём кандидата в функцию Ляпунова ${V} = {z}_{cl}\T {P}{z}_{cl} \geq 0$ и её производную от времени:
	\begin{align}
		{\dot{V}} = {z}_{cl}\T {P} {A}_{cl} {z}_{cl} +
		{z}_{cl}\T {A}_{cl}\T {z}_{cl} 
		+
		{z}_{cl}\T {P} {F}_{cl} {\zeta} +
		{\zeta}\T {F}_{cl}\T {z}_{cl}
		\leq 0
	\end{align}
	где система считается квадратичной устойчивой в конечное время тогда и только тогда, когда существует положительно-определённая матрица ${P} = \bigl[ \begin{smallmatrix}  {P}_1 & {P}_2 \\ {P}_2\T & {P}_3 \end{smallmatrix} \bigr]$ которые удовлетворяют следующим линейному матричному неравенству
	\begin{align}
		{z}_{cl}\T {P} {A}_{cl} {z}_{cl} +
		{z}_{cl}\T {A}_{cl}\T {z}_{cl} 
		+
		{z}_{cl}\T {P} {F}_{cl} {\zeta} +
		{\zeta}\T {F}_{cl}\T {z}_{cl}
		\leq 0
	\end{align}
	Перепишем как
	\begin{align}
		\begin{bmatrix}
			{P} {A}_{cl} + {A}_{cl}\T {P} & {P} {F}_{cl} \\
			{F}_{cl} \T {P} & 0
		\end{bmatrix} \leq 0.
	\end{align}
	Введём релаксацию, добавляя параметр $\alpha$, который и будет условием релаксации:
	\begin{align}\label{eq: S procedure}
		\begin{bmatrix}
			{P} {A}_{cl} + {A}_{cl}\T {P} & {P} {F}_{cl} \\
			{F}_{cl} \T {P} & 0
		\end{bmatrix} < 
		\begin{bmatrix}
			0 & 0\\
			0 & \alpha {I}
		\end{bmatrix}.
	\end{align}
	Далее нам необходимо избавиться от нелинейности в переменных и для этого вводим следующую лемму:
	\begin{lemma}
		Даны симметричные и невырожденные матрицы ${Q}_{1} \in \mathbb{R}^{n\times n}$ и ${P}_{1} \in \mathbb{R}^{n\times n}$. Следующие утверждения эквивалентны.\\
		
		i). Существуют симметричные и невырожденные матрицы  ${P}_{3}$ и ${Q}_{3} \in \mathbb{R}^{m\times m}$ и матрицы с полным рангом ${P}_{2}$ и ${Q}_{2} \in \mathbb{R}^{n\times m}$ такие, что
		
		\begin{align*}
			{P}=
			\begin{bmatrix} 
				{P}_{1} & {P}_{2}\\ 
				{P}_{2}\T & {P}_{3} 
			\end{bmatrix} =
			\begin{bmatrix} 
				{Q}_{1} & {Q}_{2} \\ 
				{Q}_{2}\T & {Q}_{3}
			\end{bmatrix}^{-1}={Q}^{-1}>0,
		\end{align*} 
		где ${Q}_{CL}=
		\begin{bmatrix} 
			{Q}_{1} & {I} \\ {Q}_{2}\T & 0
		\end{bmatrix}$ является матрицей полного ранга.\\
		
		ii). \begin{align*}
			\begin{bmatrix} 
				{Q}_{1} & I \\ 
				I & {P}_{1}
			\end{bmatrix} > 0.
		\end{align*}
	\end{lemma}
	\begin{proof}
		Из i)., берём 
		\begin{align}
			&{P} = {Q}^{-1}\\
			&{P}{Q}={I}\\
			&\begin{bmatrix} 
				{P}_{1} & {P}_{2}\\ 
				{P}_{2}\T & {P}_{3} 
			\end{bmatrix}
			\begin{bmatrix} 
				{Q}_{1} & {Q}_{2} \\ 
				{Q}_{2}\T & {Q}_{3}
			\end{bmatrix} = 
			\begin{bmatrix}
				{I} & 0 \\
				0 & {I}
			\end{bmatrix}
		\end{align}
		После последнего перемножения, мы получаем:
		\begin{align}
			&{P}_{1}{Q}_{1}+{P}_{2}{Q}_{2}\T ={I}\\
			&{P}_{1}\T{Q}_{2}+{P}_{2}{Q}_{3} =0\label{eq: lemma mult1}\\
			&{P}_{2}\T{Q}_{1}+{P}_{3}{Q}_{2}\T=0\\
			&{P}_{2}\T{Q}_{2}+{P}_{3}{Q}_{3}= {I} \label{eq: lemma mult2}
		\end{align}
		Из данных уравнений мы получаем:
		\begin{align}
			{P}{Q}_{CL}={P}_{CL},
		\end{align}
		где
		\begin{align}
			{Q}_{CL}=
			\begin{bmatrix}
				{Q}_{1} & {I}\\
				{Q}_{2}\T & 0
			\end{bmatrix}\\
			{P}_{CL}=
			\begin{bmatrix}
				{I} & {P}_{1} \\
				0 & {P}_{2}\T
			\end{bmatrix}
		\end{align}
		Матрица ${Q}_{2}$ имеет полный ранг и матрица ${Q}_{CL}$ тоже имеет полный ранг, тогда
		\begin{align}
			0<{Q}_{CL}\T {P} {Q}_{CL} = {Q}_{CL}\T {P}_{CL} = 
			\begin{bmatrix} 
				{Q}_{1} & I \\ 
				I & {P}_{1}
			\end{bmatrix}
		\end{align}
		И наоборот, предположим ii). верно. Мы можем умножить на $-1$ и применить дополнение Шура к неравенству 
		$\bigl( \begin{smallmatrix} 
			{Q}_{1} & I \\ 
			I & {P}_{1}\end{smallmatrix} \bigr) <0$. 
		И тогда, согласно дополнению Шура, получаем $-{P}_{1}<0$ и $-{Q}_{1}-(-{I})(-{P}_{1})^{-1}(-{I})\T<0$. Упрощая, избавляясь от матриц тождества, мы получаем ${P}_{1}^{-1}-{Q}_{1}<0$, и она имеет полный ранг, то есть можно найти обратную матрицу. Что также означает, что ${I}-{P}_{1}{Q}_{1}$ имеет обратную матрицу.\\\
		
		Выбираем любые две матрицы полного ранга ${P}_{2}$ и ${Q}_{2}$, такие что ${P}_{2}{Q}_{2}\T={I}-{P}_{1}{Q}_{1}$. И поскольку ${P}_2$ и ${Q}_2$ имеют полный ранг, ${Q}_{CL}=\bigl[\begin{smallmatrix}  
			{Q}_{1} & {I}\\
			{Q}_{2}\T & 0
		\end{smallmatrix} \bigr]$ и 
		${P}_{CL}=\bigl[ \begin{smallmatrix}
			{I} & {P}_{1} \\
			0 & {P}_{2}\T
		\end{smallmatrix} \bigr]$ также имеют полный ранг и существуют обратные матрицы к ним.\\
		
		Обозначим ${P}$ и ${Q}$ как 
		\begin{align*}
			{P} = {P}_{CL}{Q}_{CL}^{-1} \qquad и \qquad {Q} ={Q}_{CL}{P}_{CL}^{-1}.
		\end{align*} 
		Далее умножая ${P}$ и ${Q}$, мы получаем единичную матрицу
		\begin{align*}
			{P}{Q}={P}_{CL}{Q}_{CL}^{-1}{Q}_{CL}{P}_{CL}^{-1}={I}.
		\end{align*}
		Таким образом ${P}={Q}^{-1}$.
		${P}_3$ и ${Q}_3$ при необходимости можно найти из (\ref{eq: lemma mult1}-\ref{eq: lemma mult2}).
	\end{proof}
	Домножим с правой и левой стороны правую часть (\ref{eq: S procedure}) на матрицы $diag({Q}_{CL}\T, {I}, {I})$ и $diag({Q}_{CL}, {I}, {I})$ для линеаризации неравенств в переменных:
	\begin{align}
		\begin{bmatrix}
			{Q}_{CL}\T{A}_{CL}\T {P}_{CL} + {P}_{CL}\T{A}_{CL}{Q}_{CL} & {P}_{CL}\T{F}_{CL} \\
			{F}_{CL}\T{P}_{CL} & 0
		\end{bmatrix} < 
		\begin{bmatrix}
			0 & 0\\
			0 & \alpha {I}
		\end{bmatrix}.
	\end{align}
	Далее мы постепенно раскрываем и упрощаем каждый компонент уравнения
	\begin{align}
		{P}_{CL}\T{F}_{CL} = \begin{bmatrix}
			{I} & 0 \\
			{P}_{1} & {P}_{2}
		\end{bmatrix}
		\begin{bmatrix}
			{A}_R + {B}{D}_K{C}{R}\\ {B}_K{C}_R
		\end{bmatrix} =
		\begin{bmatrix}
			{A}_R + {B}{D}_K{C}{R}\\
			{P}_{1}{A}_R + {P}_{1}{B}{D}_K{C}{R}+{P}_1{B}_K{C}{R}
		\end{bmatrix}
	\end{align}
	%
	\begin{align}
		&{Q}_{CL}\T{A}_{CL}\T {P}_{CL} + {P}_{CL}\T{A}_{CL}{Q}_{CL} = \nonumber \\
		& \begin{bmatrix}
			{A}_N{Q}_{1} +{Q}_{1}{A}_N\T + {B}{\hat{C}}_K +{\hat{C}}_K\T{B}\T & {A}_N + {\hat{A}}_K\T +{B}{D}_K{C}{N}\\
			{A}_N\T + {N}\T{C}\T{D}_K\T{B}\T+{\hat{A}}_K & {P}_{1}{A}_N +{A}_N\T{P}_{1}+{\hat{B}}_K{C}{N}+{N}\T{C}\T{\hat{B}}_K\T
		\end{bmatrix}   
	\end{align}
	\begin{align}
		{\hat{C}}_K= {D}_K{C}{N}{Q}_{1} + {C}_K{Q}_{2}\T\\
		{\hat{B}}_K = {P}_{1}{B}{D}_K + {P}_{2}{B}_K\\
		{\hat{A}}_K = {P}_{1}({A}_N+{B}{D}_K{C}{N}){Q}_{1}+{P}_{2}{B}_K{C}{N}{Q}_{1}+{P}_{1}{B}{C}_K{Q}_{2}\T+{P}_{2}{A}_K{Q}_{2}\T
	\end{align}
	Финальное неравенство
	\begin{align}\label{eq: final LMI}
		\begin{bmatrix}
			{\Theta}_1  & {A}_N + {\hat{A}}_K\T +{B}{D}_K{C}{N} & {A}_R + {B}{D}_K{C}{R}\\
			\cdots & {\Theta}_2 & {P}_{1}{A}_R + {P}_{1}{B}{D}_K{C}{R}+{P}_1{B}_K{C}{R}\\
			\cdots & \cdots & 0 
		\end{bmatrix} < 
		\begin{bmatrix}
			0 & 0\\
			0 & \alpha {I}
		\end{bmatrix}
	\end{align}
	где
	\begin{align}
		{\Theta}_1 = {A}_N{Q}_{1} +{Q}_{1}{A}_N\T + {B}{\hat{C}}_K +{\hat{C}}_K\T{B}\T \\
		{\Theta}_2 = {P}_{1}{A}_N +{A}_N\T{P}_{1}+{\hat{B}}_K{C}{N}+{N}\T{C}\T{\hat{B}}_K\T
	\end{align}
	И держится неравенство: 
	\begin{align}\label{eq: condition}
		\begin{bmatrix} 
			{Q}_{1} & I \\ 
			I & {P}_{1}
		\end{bmatrix} > 0.
	\end{align}
	Это и является линейным матричным неравенством с переменными ${Q}_1,{P}_1, {\hat{A}}_K, {\hat{B}}_K,{\hat{C}}_K, {D}_K$ и $\alpha$.
\end{proof}
Целевой функцией будет являться минимизация параметра  $\alpha$ (добавить почему). Оптимизационная задача будет выглядеть следующим образом:
%
\begin{equation}
	\label{eq:thm2_OCP}
	\begin{aligned}
		& \underset{{Q}_1,{P}_1, {\hat{A}}_K, {\hat{B}}_K,{\hat{C}}_K, {D}_K, \alpha }{\text{минимизируя}}
		& &  \alpha, \\
		& \text{при ограничениях}
		& & \begin{cases}
			\text{условия \eqref{eq: final LMI}, \eqref{eq: condition}}.
		\end{cases}
	\end{aligned}
\end{equation}
После решения оптимизационной задачи, мы можем найти коэффициенты следующим образом:
\begin{align}
	{P}_{2} = ({I}-{P}_{1}{Q}_{1})({Q}_{2}\T)^{-1}\\
	{C}_K = ({\hat{C}}_K - {D}_K {C} {N} {Q}_{1} ) ({Q}_{2}\T)^{-1}\\
	{B}_K = {P}_{2}^{-1}({\hat{B}}_K- {P}_{1}{B}{D}_K)\\
	{A}_K = {P}_{2}^{-1}({\hat{A}}_K-{P}_{1}({A}_N+{B}{D}_K{C}{N}){Q}_{1} - {P}_{2}{B}_K{C}{C}{Q}_{1}-{P}_{1} {B}{C}_K{Q}_{2}\T)({Q}_{2}\T)^{-1}
\end{align}

\section{Эксперименты}\label{sec:ch5/sect2}
Сначала мы продемонстрируем работу предложенных методов. Мы решаем оптимизационную задачу \eqref{eq:thm1_OCP} для случая смешанной неопределённости (аддитивной и мультипликативной) и оптимизационную задачу \eqref{eq:thm3_OCP} только для мультипликативной неопределённости. Мы решаем обе задачи для различных значений $\epsilon_1$. На рисунке \ref{fig:cost} показано, как их соответствующие оптимальные затраты зависят от выбора $\epsilon_1$. Результаты показывают, что задача выполнима, а оптимальная стоимость как функция от $\epsilon_1$ описывает выпуклую кривую, что упрощает выбор оптимального значения этого параметра.

Во-вторых, мы показываем, как предложенные методы могут быть использованы для поиска наибольшего структурированного набора неопределённостей, которые может выдержать робастный линейный регулятор. Такая задача имеет ряд практических приложений. Она может быть использована для определения того, достигла ли конструкция регулятора своих пределов с точки зрения неопределённостей, которые он должен переносить. Задача также может быть решена для определения того, насколько велик набор неопределённостей, которые могут быть допустимы для данного конкретного робота и его конкретной конфигурации, что облегчает соответствующий анализ.

\subsection{Параметры эксперимента}\label{sec:ch5/sect2/sub1}
\subsection{Результаты эксперимента}\label{sec:ch5/sect2/sub2}