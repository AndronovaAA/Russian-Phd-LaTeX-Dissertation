\chapter{Обзор литературы}\label{ch:ch1}

\section{Шагающие роботы}\label{sec:ch1/sec1}
\section{Управление для шагающих роботов}\label{sec:ch1/sec2}
Управление шагающими роботами представляет собой сложную задачу, которая решается с помощью различных наборов систем, включая навигацию и низкоуровневые контуры управления отдельными приводами робота. В частности, в рамках общей задачи управления шагающим роботом можно выделить две ключевые подзадачи: планирование движения на высоком уровне и планирование движения низкого уровня \cite{WalkingRobots}. В первой подзадаче стоит вопрос "куда двигаться", а вторая отвечает на вопрос "как двигаться" в направлении заданном в первой подзадаче.

При изучении управления для шагающих роботов существует множество направлений исследований. Остановимся на самых значимых.



\subsection{Линеаризация}\label{sec:ch1/sec2/sub5}
Линеаризовать динамику шагающего робота можно, взяв линейный член разложения Тейлора его нелинейной модели. Проецируя его динамику на касательное пространство к множеству ограничений (множество допустимых состояний при текущем контакте с окружающей средой), мы получаем линейную стационарную модель, которая может быть использована, например, для проектирования линейно-квадратичного регулятора \cite{mason2014full}. Однако при этом не учитывается эффект статических состояний, о котором говорилось ранее. Выделение статических состояний и рассмотрение их как постоянной составляющей динамики позволяет спроектировать наблюдатель состояния как для активных, так и для статических состояний, как это было сделано в \cite{SAVIN2021}. Этот подход все ещё ограничен случаями, когда мультипликативная неопределённость отсутствует. В данной работе мы предлагаем метод, способный работать как со статическими состояниями, так и с мультипликативной неопределённостью.

Отдельно отметим, что существует целый ряд работ по проектированию управления для очень большого класса систем, описываемых ДАУ или вырожденными ОДУ, называемых дескрипторными системами. В этой области исследований есть результаты по робастному управлению и проектированию наблюдателей \cite{Cheng2018, Darouach2014}. Однако методы, разработанные в этой области, недостаточны для решения проблемы, рассматриваемой в данной статье.

\section{Ортогональные методы}\label{sec:ch1/sec3}
Метод оценки состояния, предложенный в данной работе, основан на ортогональных проекциях.Существует ряд существующих методов, основанных на ортогональных проекциях и декомпозиции, разработанных для систем с явными ограничениями, которые мы рассмотрим в этом подразделе. Результаты, наиболее близкие к предлагаемому в данной работе, можно найти в работах, посвящённых управлению механическими системами с явными ограничениями, поэтому мы сосредоточим наше обсуждение на них.

В работе \cite{Aghili2003} была предложена процедура для моделирования механических систем с явными ограничениями, что устанавливает связь с более ранними работами в области численных методов для ДАУ, такими как \cite{Liang1987} и другие, в которых рассматривалась очень похожая проблема.

\section{Робастное управление}\label{sec:ch1/sec4}
Оценка состояния необходима, когда значения некоторых переменных состояния, требуемых законом управления не доступны непосредственно из измерений. Это может быть результатом отсутствия датчиков, шума измерений или выбранной параметризации системы. В типичном сценарии закон управления требует полной информации о состоянии, в то время как только некоторые переменные состояния непосредственно измеряются.

В литературе предложено множество методов проектирования робастных регуляторов и наблюдателей. Например, методы, основанные на линейно-матричных неравенствах, были предложены для проектирования робастных регуляторов для линейных стационарных систем с нормированными мультипликативными неопределённостями \cite{POLYAK2021,ROTONDO2014}.
Одновременное проектирование регулятора и наблюдателя решается как серия задач линейных матричных неравенств \cite{ZEMOUCHE2015,GRITLI2021}. Были предложены $H_\infty$ методы проектирования регуляторов и наблюдателей для линейных стационарных систем с $L_2$ нормированными аддитивными неопределённостями \cite{Bennani2019, KHELOUFI2016}.

В робототехнике связанной с шагающими роботами к робастной устойчивости подходили с разных сторон, включая стратегии восстановления после толчка \cite{Pratt2006}, разработку управления с прогнозирующими моделями со свойствами робастной устойчивости \cite{KIM2019} и т.д. Несмотря на то, что были достигнуты значительные результаты в плане производительности роботов, не хватало методов с формальными гарантиями робастности. Прямое применение линейных методов управления к управлению шагающими роботами было затруднено из-за свойств динамики таких роботов, которые приводят к другому типу линейных моделей, о чем будет рассказано в следующем подразделе.

\section{Робастность и мультипликативная неопределённость в модели}\label{sec:ch1/sec5}
Использование теории Ляпунова для одновременного проектирования линейного регулятора и наблюдателя Люнбергера для линейной стационарной системы приводит к оптимизационным задачам с ограничениями в виде билинейных матричных неравенств. В \cite{LIEN2004} эта проблема решается путём добавления дополнительного условия для преобразования задачи к линейно-матричному неравенству; там авторы рассматривали неопределённости как в матрице состояния, так и в матрице управления линейной стационарной системы. В \cite{KHELOUFI2013} авторы использовали неравенство Юнга для преодоления проблемы билинейности, и их метод также учитывает неопределённости как в матрице состояния, так и в матрице наблюдения. В работе \cite{ZEMOUCHE2015} предложен двухшаговый алгоритм для решения проблемы ограничений билинейных матричных неравенств и неопределённостей в матрицах состояния и наблюдения. Неравенство Юнга было использовано для облегчения проектирования управления на основе наблюдателей, а также для работы с мультипликативными неопределённостями, связанными с нормой. В качестве альтернативы в \cite{GRITLI2021} авторы предлагают метод линейных матричных неравенств для работы с неопределённостями в матрицах состояния, управления и наблюдения, основанный на двух оригинальных леммах; однако численные трудности, связанные с поиском сетки по значению свободного параметра, оказываются одинаковыми во всех приведённых методах.
 
\section{Используемые математические инструменты}\label{sec:ch1/sec6}
Неравенство Юнга, а также S-процедура \cite{Amato2011,LIEN2008} способствуют возникновению свободных скалярных параметров, которые могут внести нелинейность в задачу, особенно когда в ограничениях присутствует как переменная, так и ее инверсия. Решение этой проблемы было описано в \cite{KHELOUFI2016}, где один положительный скаляр и его инверсия были заменены двумя независимыми положительными скалярами. Другая проблема возникает, когда эти скалярные параметры вводят билинейность в задачу. Для этого случая в \cite{KHELOUFI2013} был предложен метод решетчатого поиска, позволяющий превратить одну билинейную задачу в серию выпуклых задач с ограничениями в виде линейных матричных неравенств.

В данной работе мы используем неравенство Юнга для решения проблемы билинейности; неопределённости в матрицах моделей также обрабатываются с помощью неравенства Юнга и S-процедуры.
\section{Ближайшие аналоги}\label{sec:ch1/sec7}
\FloatBarrier
