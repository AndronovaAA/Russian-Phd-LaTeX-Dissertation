\chapter{Обзор литературы}\label{ch:ch1}

\section{Шагающие роботы}\label{sec:ch1/sec1}

Роботы на ногах предлагают более широкие возможности по сравнению с колёсными и гусеничными роботами с точки зрения условий работы. Ножные роботы могут передвигаться по обычной и неровной местности без каких-либо аппаратных изменений и демонстрируют исключительную мобильность \cite{Silva2012}. Колёсные транспортные средства требуют для передвижения асфальтированных (или хотя бы обычных) поверхностей, при этом они чрезвычайно быстры и эффективны на них. В то же время эти механизмы могут быть простыми и лёгкими. Однако более 50\% поверхности Земли недоступно для традиционных транспортных средств, и колёсным машинам трудно или даже невозможно справиться с большими препятствиями и неровностями поверхности. Даже вездеходы могут преодолевать лишь небольшие препятствия и неровности поверхности, но только за счёт высокого потребления энергии \cite{Bekker1962}.  Что касается гусеничных машин, то, хотя они и обеспечивают повышенную мобильность на сложных участках, они не в состоянии преодолеть многие трудности, а их энергопотребление относительно велико.  К этим проблемам следует добавить тот факт, что традиционные транспортные средства оставляют на земле сплошные колеи, что в некоторых ситуациях невыгодно, например, с экологической точки зрения. 

Из всего вышесказанного можно сделать вывод, что системы передвижения на ногах обеспечивают лучшую мобильность на естественных участках местности, поскольку эти транспортные средства могут использовать отдельные опоры для каждой ноги, в отличие от колёсных транспортных средств, которым необходима непрерывная опорная поверхность. Таким образом, эти транспортные средства могут передвигаться по неровной местности, изменяя конфигурацию ног, чтобы адаптироваться к неровностям поверхности, и, кроме того, ноги могут устанавливать контакт с землёй в отдельных точках в соответствии с условиями местности. По этим причинам ноги по своей сути являются подходящими системами для передвижения по неровной поверхности. Когда транспортные средства передвигаются по мягким поверхностям, например, по песчаной почве, возможность использовать дискретные опоры в грунте может также улучшить энергопотребление, поскольку они деформируют рельеф меньше, чем колёсные или гусеничные транспортные средства \cite{Iida2007}. 

Поэтому энергия, необходимая для выхода из впадин, меньше \cite{Bekker1962}; \cite{bekker1969}, а площадь контакта стопы с землёй может быть такой, чтобы давление на опору было небольшим. Кроме того, использование нескольких степеней свободы в суставах ног позволяет шагающим роботам менять направление движения без проскальзывания. Также можно варьировать высоту корпуса, создавая эффект демпфирования и развязки между неровностями рельефа и кузовом транспортного средства (и, как следствие, его полезной нагрузкой). Что касается движения, то следует также упомянуть о возможности, которую предоставляют эти системы, прижиматься к местности, по которой они движутся. Это особенно верно, если они перемещаются, например, по внешней поверхности труб, чтобы повысить способность к балансировке \cite{Kaneko2002}.

Несмотря на все потенциальные преимущества шагающих роботов, на текущем этапе развития есть несколько аспектов, которые необходимо улучшить и оптимизировать.

Четвероногие роботы - лучший выбор среди всех ножных роботов с точки зрения мобильности и стабильности локомоции. Четыре ноги робота легко контролируются, проектируются и обслуживаются по сравнению с двумя или шестью ногами. Биологически вдохновлённая локомоция беговых походки способна выдержать большую полезную нагрузку и балансировку робота-четвероногого робота, начатую исследователем ранее. Для достижения скорости в реальном времени и естественного движения, как у коровы, собаки, гепарда, необходима разработанная система управления и динамическая генерация походки четвероногих роботов \cite{gonccalves2013}.

Четырёхногие роботы вдохновлены локомоцией четвероногих животных: шарнирные соединения в коленях, бёдрах и шее позволяют им повторять движения животных. Самый ранний пример такого робота датируется 1870 годом, когда Чебышев сконструировал четвероногий механизм, который мог стоять и ходить, но не мог адаптироваться к местности. В 1893 году Райгг получил первый патент на ножную машину - симулятор верховой езды, приводимый в движение педалями \cite{patent}. Заметный шаг в развитии четвероногих роботов произошёл в 1940 году, когда Хатчинсон сконструировал автономно управляемого робота на ногах, подчёркивая преимущества педальных систем перед колёсными или гусеничными машинами для работы с тяжёлыми грузами \cite{Hutchinson1967}.  Кульминацией этого подхода стала постройка компанией Bucyrus-Eire в 1969 году самой большой машины на ногах Big Muskie. Этот массивный робот проработал в открытых карьерах 22 года, продемонстрировав целесообразность использования четвероногих для крупномасштабных промышленных применений.

В 1980-х годах профессор Хиросэ из Токийского технологического института создал семейство четвероногих роботов, ключевой вехой которого стал PV-II. Этот робот оснащён моторизованным приводом для каждой ноги, что позволяет ему эффективно ходить\cite{Hirose2009}. Дальнейшие разработки, такие как четвероногий робот OSQ из Стэнфорда, продвинули понимание галопирующей походки и энергоэффективной локомоции \cite{Nichol2004}.  Совсем недавно китайские институты, такие как Шаньдунский университет, разработали четвероногих роботов SCalf-1 и SCalf-2, оснащённых усовершенствованными гидравлическими системами для улучшения адаптации к местности и повышения скорости \cite{Rong2012}.

\section{Управление для шагающих роботов}\label{sec:ch1/sec2}

При рассмотрении системы четвероногих роботов как плавающей в пространстве системы с несколькими телами исследования в основном были сосредоточены на определении положения и ориентации основания, а также каждой ноги. Переменные состояния включают в себя конфигурацию тела и углы сочленения ног. Управляющие силы в системе включают крутящие моменты в суставах и силы реакции на грунт.

Желаемая траектория движения тела и каждой ноги может быть сформирована с помощью предварительного планирования или ограничений. Помимо ограничений, связанных с контактом с землёй и трением,\cite{henze2017multi} задачи могут быть описаны в виде уравнений или уравнений неравенств, включающих переменные состояния или моменты вращения суставов. Чтобы решить проблему противоречивых ограничений задачи, необходимо использовать соответствующие уравнения оптимизации при проектировании иерархического управления. Такой иерархический контроллер, известный как регулятор всего тела,\cite{fahmi2019passive} объединяет задачи для всех систем робота. 

Планирование и управление движениями занимают центральное место в работе четвероногих роботов, включая формирование и выполнение походки.  К распространённым методам формирования походки относятся генераторы центрального паттерна (ГЦП), модель перевёрнутого маятника с пружинной нагрузкой (ППМ), метод нулевой точки момента (НМП) и траектории кривых Безье.

Управление шагающими роботами представляет собой сложную задачу, которая решается с помощью различных наборов систем, включая навигацию и низкоуровневые контуры управления отдельными приводами робота. В частности, в рамках общей задачи управления шагающим роботом можно выделить две ключевые подзадачи: планирование движения на высоком уровне и планирование движения низкого уровня \cite{WalkingRobots}. В первой подзадаче стоит вопрос "куда двигаться", а вторая отвечает на вопрос "как двигаться" в направлении заданном в первой подзадаче.

При изучении управления для шагающих роботов существует множество направлений исследований. Остановимся на самых значимых.

Кроме того, ограничения вводят замкнутые кинематические цепи, делая системы чрезмерно управляемыми; с другой стороны, в отсутствие ограничений такие системы, как шагающие роботы с плавающей базой, являются недостаточно управляемыми и неконтролируемыми. Это создаёт дополнительные проблемы с оценкой состояния, поскольку некоторые состояния в координатах плавающей базы не наблюдаются и не влияют на динамику системы. Более того, структура динамики механических систем приводит к тому, что одни и те же переменные появляются как в состоянии, так и в его производной по времени (это происходит с обобщёнными скоростями в динамике Лагранжа или уравнениями манипулятора, когда они выражаются в форме пространства состояний). Эта информация может быть потеряна, если ограничения накладываются только на производные состояния, как это было сделано в \cite{Mason2017}; если она учитывается, это дополнительно упрощает задачу оценивания состояния

Модельное прогнозирующее управление - это метод итеративного решения проблем оптимизации на основе режимов, учитывающий текущее состояние системы и прогнозирующий его изменение в будущем.

\subsection{Линеаризация}\label{sec:ch1/sec2/sub5}
Линеаризовать динамику шагающего робота можно, взяв линейный член разложения Тейлора его нелинейной модели. Проецируя его динамику на касательное пространство к множеству ограничений (множество допустимых состояний при текущем контакте с окружающей средой), мы получаем линейную стационарную модель, которая может быть использована, например, для проектирования линейно-квадратичного регулятора \cite{mason2014full}. Однако при этом не учитывается эффект статических состояний, о котором говорилось ранее. Выделение статических состояний и рассмотрение их как постоянной составляющей динамики позволяет спроектировать наблюдатель состояния как для активных, так и для статических состояний, как это было сделано в \cite{SAVIN2021}. Этот подход все ещё ограничен случаями, когда мультипликативная неопределённость отсутствует. В данной работе мы предлагаем метод, способный работать как со статическими состояниями, так и с мультипликативной неопределённостью.

Отдельно отметим, что существует целый ряд работ по проектированию управления для очень большого класса систем, описываемых ДАУ или вырожденными ОДУ, называемых дескрипторными системами. В этой области исследований есть результаты по робастному управлению и проектированию наблюдателей \cite{Cheng2018, Darouach2014}. Однако методы, разработанные в этой области, недостаточны для решения проблемы, рассматриваемой в данной статье.

\section{Ортогональные методы}\label{sec:ch1/sec3}
Метод оценки состояния, предложенный в данной работе, основан на ортогональных проекциях. Существует ряд существующих методов, основанных на ортогональных проекциях и декомпозиции, разработанных для систем с явными ограничениями, которые мы рассмотрим в этом подразделе. Результаты, наиболее близкие к предлагаемому в данной работе, можно найти в работах, посвящённых управлению механическими системами с явными ограничениями, поэтому мы сосредоточим наше обсуждение на них.

В работах \cite{Aghili2003},\cite{Aghili2005}, а также в более поздних работах \cite{Mistry2010},\cite{Righetti2011},\cite{Righetti2013} были использованы ортогональные проекции. В \cite{Aghili2003},\cite{Aghili2005} был предложен ряд новых эквивалентных моделей в неминимальных координатах, в то время как в \cite{Mistry2010},\cite{Righetti2011},\cite{Righetti2013}, следуя более ранним исследованиям \cite{Khatib2007},\cite{Sentis2005}, основное внимание было уделено разработке закона управления. Однако, как было показано в \cite{Righetti2011}, ряд независимо разработанных проекционных методов управления дает одинаковые (вплоть до разрешения избыточности момента) законы управления.

В работе \cite{Aghili2003} была предложена процедура для моделирования механических систем с явными ограничениями, что устанавливает связь с более ранними работами в области численных методов для ДАУ, такими как \cite{Liang1987} и другие, в которых рассматривалась очень похожая проблема.

\section{Робастное управление}\label{sec:ch1/sec4}
Оценка состояния необходима, когда значения некоторых переменных состояния, требуемых законом управления не доступны непосредственно из измерений. Это может быть результатом отсутствия датчиков, шума измерений или выбранной параметризации системы. В типичном сценарии закон управления требует полной информации о состоянии, в то время как только некоторые переменные состояния непосредственно измеряются.

В литературе предложено множество методов проектирования робастных регуляторов и наблюдателей. Например, методы, основанные на линейно-матричных неравенствах, были предложены для проектирования робастных регуляторов для линейных стационарных систем с нормированными мультипликативными неопределённостями \cite{POLYAK2021,ROTONDO2014}.
Одновременное проектирование регулятора и наблюдателя решается как серия задач линейных матричных неравенств \cite{ZEMOUCHE2015,GRITLI2021}. Были предложены $H_\infty$ методы проектирования регуляторов и наблюдателей для линейных стационарных систем с $L_2$ нормированными аддитивными неопределённостями \cite{Bennani2019, KHELOUFI2016}.

В робототехнике связанной с шагающими роботами к робастной устойчивости подходили с разных сторон, включая стратегии восстановления после толчка \cite{Pratt2006}, разработку управления с прогнозирующими моделями со свойствами робастной устойчивости \cite{KIM2019} и т.д. Несмотря на то, что были достигнуты значительные результаты в плане производительности роботов, не хватало методов с формальными гарантиями робастности. Прямое применение линейных методов управления к управлению шагающими роботами было затруднено из-за свойств динамики таких роботов, которые приводят к другому типу линейных моделей, о чем будет рассказано в следующем подразделе.

\section{Робастность и мультипликативная неопределённость в модели}\label{sec:ch1/sec5}
Использование теории Ляпунова для одновременного проектирования линейного регулятора и наблюдателя Люнбергера для линейной стационарной системы приводит к оптимизационным задачам с ограничениями в виде билинейных матричных неравенств. В \cite{LIEN2004} эта проблема решается путём добавления дополнительного условия для преобразования задачи к линейно-матричному неравенству; там авторы рассматривали неопределённости как в матрице состояния, так и в матрице управления линейной стационарной системы. В \cite{KHELOUFI2013} авторы использовали неравенство Юнга для преодоления проблемы билинейности, и их метод также учитывает неопределённости как в матрице состояния, так и в матрице наблюдения. В работе \cite{ZEMOUCHE2015} предложен двухшаговый алгоритм для решения проблемы ограничений билинейных матричных неравенств и неопределённостей в матрицах состояния и наблюдения. Неравенство Юнга было использовано для облегчения проектирования управления на основе наблюдателей, а также для работы с мультипликативными неопределённостями, связанными с нормой. В качестве альтернативы в \cite{GRITLI2021} авторы предлагают метод линейных матричных неравенств для работы с неопределённостями в матрицах состояния, управления и наблюдения, основанный на двух оригинальных леммах; однако численные трудности, связанные с поиском сетки по значению свободного параметра, оказываются одинаковыми во всех приведённых методах.
 
\section{Используемые математические инструменты}\label{sec:ch1/sec6}
Неравенство Юнга, а также S-процедура \cite{Amato2011,LIEN2008} способствуют возникновению свободных скалярных параметров, которые могут внести нелинейность в задачу, особенно когда в ограничениях присутствует как переменная, так и ее инверсия. Решение этой проблемы было описано в \cite{KHELOUFI2016}, где один положительный скаляр и его инверсия были заменены двумя независимыми положительными скалярами. Другая проблема возникает, когда эти скалярные параметры вводят билинейность в задачу. Для этого случая в \cite{KHELOUFI2013} был предложен метод решетчатого поиска, позволяющий превратить одну билинейную задачу в серию выпуклых задач с ограничениями в виде линейных матричных неравенств.

В данной работе мы используем неравенство Юнга для решения проблемы билинейности; неопределённости в матрицах моделей также обрабатываются с помощью неравенства Юнга и S-процедуры.
\section{Ближайшие аналоги}\label{sec:ch1/sec7}
\FloatBarrier
