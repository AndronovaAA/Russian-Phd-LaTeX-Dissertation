\section{Введение}

\subsection{Актуальность исследования}

\begin{frame}
	\frametitle{Шагающие роботы}
	    \begin{minipage}[t]{0.47\linewidth}
	    	\begin{figure}
			\includegraphics[width=\textwidth]{unitree.png}
			\caption{Четырёхногий робот Unitree A1}
			\end{figure}
	\end{minipage}
	\hfill
	\begin{minipage}[t]{0.47\linewidth}
		Математическая модель для описания динамики шагающего робота:
		\begin{itemize}
			\item С явными механическими ограничениями;
			\item Линеаризованная;
			\item С параметрическими неопределённостями.
		\end{itemize}		
	\end{minipage}
\end{frame}

\begin{frame}
    \frametitle{Управление шагающими роботами}
	\begin{figure}
	\centering
	\includegraphics[scale=0.59]{images/table.JPG}
	\caption{Методы управления и наблюдения для систем с механическими ограничениями.}
	\label{fig:table}
\end{figure}
\end{frame}

\subsection{Область исследования}

\begin{frame}
    \frametitle{Предмет и объект исследования}
	Объектом исследования является оптимальное робастное управление шагающими роботами с параметрическими неопределённостями.
	\vfill
	Предметом исследования являются методы математического моделирования системы с неопределённостями, описывающие динамику шагающего робота и численные методы с использованием ортогональных методов и линейных матричных неравенств для нахождения коэффициентов регулятора и наблюдателя с последующими вычислениями при помощи комплекса программ.
\end{frame}

\subsection{Цели и задачи}

\begin{frame}
    \frametitle{Цели и задачи}
    Целью диссертационной работы является разработка численных ортогональных методов оптимального робастного управления математическими моделями, описывающих шагающих роботов, с использованием линейных матричных неравенств.
    
	Выполненные задачи:
	\begin{itemize}
		\item Проведён обзор литературы;
		\item Описана математическая модель шагающего робота;
		\item Предложен численный метод оптимального робастного управления для систем с параметрическими неопределённостями;
		\item Предложен численный метод оптимального робастного управления с динамической обратной связью для системы с аддитивными неопределённостями;
		\item Разработан комплекс программ, реализующий предложенные методы.
	\end{itemize}
\end{frame}

\section{Основная часть}

\subsection{Математическое моделирование}

\begin{frame}
    \frametitle{Ортогональная декомпозиция}
    Конфигурация шагающего робота может быть описана вектором обобщённых координат $q$. Обобщённые координаты и их производные $\dot{q}$ вместе образуют вектор состояния робота $x$. \\
    
    Взаимодействие робота с окружающей средой может быть описано как ограничение на скорость изменения состояния: ${G}\dot{x}= 0$. Это мотивирует ортогональную декомпозицию вектора состояния:
    \begin{align}\label{eq:1}
    	{x}= {N} {z}^* + {R} {\zeta}^*
    \end{align}
    где ${z}^*$ --- координаты состояния, изменяющиеся со временем, и ${\zeta}^* = const$; матрицы ${N}$ и ${R}$ являются ортонормированными базисами в пространстве нулей и строк матрицы ограничений ${G}$.
\end{frame}

\begin{frame}
	\frametitle{Линеаризация}
	Рассмотрим динамику $\dot z^* = f(z^*, \zeta^*)$ с параметрами $\zeta^* = \text{const}$ и ее разложение Тейлора в окрестности точки $z^* = z_0$, $\zeta^* = \zeta_0$. Обозначим $z = z^* - z_0$ и $\zeta = \zeta^* - \zeta_0$.
	Опустив квадратичные члены и члены высших порядков, но сохранив линейные и билинейные члены, запишем приближенную динамику в виде:
	
	\begin{equation}
		\label{eq:model_linearization_1}
		\dot z = \frac{\partial f}{\partial z} z + \Delta {A}(\zeta) z + \frac{\partial f}{\partial \zeta} \zeta,
	\end{equation}
	где $\Delta {A}(\zeta) = \frac{\partial}{\partial z} (\frac{\partial f}{\partial \zeta} \zeta)$.
	
	\vfill
	
	Модель \eqref{eq:model_linearization_1} включает в себя как мультипликативные, так и аддитивные неопределённости. Мы можем интерпретировать постоянные параметры $\zeta$ как неопределённые параметры модели, либо как статические компоненты состояния шагающего робота. В последнем случае мы приходим к модели шагающего робота на временном интервале между последовательными шагами.
\end{frame}

\subsection{Методы оптимального робастного управления}

\begin{frame}
    \frametitle{Система с мультипликативными неопределённостями}
    Рассмотрим следующую систему:
    \begin{equation}
    	\label{eq:part2_linear_dynamics}
    	\begin{cases}
    		\dot z=({A}_n+\Delta {A}_n)z + ({B}+\Delta {B})u,\\
    		y = ({C}+ \Delta {C}){N}  z,
    	\end{cases}
    \end{equation}
    где $A_n \in \mathbb{R}^{n_z \times n_z}$ --- матрица состояния, $B \in \mathbb{R}^{n_z \times m}$ --- матрица управления, $C \in \mathbb{R}^{l \times n}$ --- матрица наблюдения, а неопределённые матрицы состояния, управления и наблюдения определяются как:
    %
    \begin{equation}
    	\label{eq:part2_uncertainty}
    	\Delta {A}_n={M}_1{F}_1{N}_1, \ \ \Delta {B}= {M}_2{F}_2{N}_2, \ \
    	\Delta {C} = {M}_3{F}_3{N}_3,
    \end{equation}
    %
    где ${M}_1 \in \mathbb{R}^{n_z \times d}$, 
    ${N}_1 \in \mathbb{R}^{d \times n_z}$ , ${M}_2 \in \mathbb{R}^{n_z \times p}$,
    ${N}_2 \in \mathbb{R}^{p \times m}$, ${M}_3 \in \mathbb{R}^{l \times q}$,
    ${N}_3 \in \mathbb{R}^{q \times n}$ --- известные матрицы, 
    а ${F}_i$ --- неизвестные матрицы ограниченные по норме ${F}_i\T{F}_i\leq \nu_i {I}$, где $\nu_i$ --- радиусы неопределённости.
    
    Вводим наблюдатель Люенбергера с состоянием $\hat{z}$:
    %
    \begin{equation}
    	\label{eq:Luenberger}
    	\dot{\hat{z}}={A}_n\hat{z}+{B}u+{L}_z(y- {C} {N}\hat{z}),
    \end{equation}
    %
    где ${L}_z$ --- матрица наблюдателя. 
 
\end{frame}

\begin{frame}
	\frametitle{Система с мультипликативными неопределённостями}
	    Рассмотрим следующий закон управления с линейной обратной связью:
	%
	\begin{equation}
		\label{eq:control_law}
		u={K}_z\hat{z}.
	\end{equation}
	После введения ошибки как $e_z=z-\hat{z}$, получаем систему:
	\begin{equation}
		\label{eq:part2_system}
		\begin{bmatrix}
			\dot{z} \\ \dot{e}_z
		\end{bmatrix}=\begin{bmatrix}
			({A}_n+\Delta {A}_n +{B}{K}_z+\Delta {B}{K}_z) & -({B}{K}_z+\Delta {B}{K}_z) \\
			(\Delta {A}_n +\Delta {B}{K}_z-{L}_z\Delta {C}{N}) & ({A}_n-{L}_z{C}{N}-\Delta {B}{K}_z)        \end{bmatrix}\begin{bmatrix}
			z \\ e_z
		\end{bmatrix}.
	\end{equation}
	\begin{block}{Теорема 1}
		Система (\ref{eq:part2_system}) асимптотически стабильна для всех матриц
		$\Delta {A}_n={M}_1{F}_1{N}_1$, 
		$\Delta {B}= {M}_2{F}_2{N}_2$, 
		$\Delta {C} = {M}_3{F}_3{N}_3$,
		где
		${F}_i\T{F}_i\leq \nu_i {I}$ для $i=1,2,3$,
		если существуют такие положительно определённые матрицы ${Q}_1$, ${P}_2$ и положительные скаляры
		$\gamma_1>0$, $\gamma_2>0$, $\gamma_3>0$, $\bar{\mu}_1>0$, $\bar{\mu}_2>0$, $\bar{\mu}_3>0$, и $\epsilon_1 > 0$ такие, чтобы следующее линейное матричное неравенство выполнялось:
	\end{block}
\end{frame}

\begin{frame}
	\frametitle{Система с мультипликативными неопределённостями}
	\begin{block}{Теорема 1. Продолжение}
	\[
\label{eq:lmi1}
\begin{medsize}
	\begin{bmatrix}
		{\Lambda}_1 & 0 & {M}_1 & {M}_2&0 & {Q}_1{N}_1\T & {\Lambda}_3 &{\Lambda}_4 & {B}\hat{{K}}_z & 0\\
		* & {\Lambda}_2 & {P}_2{M}_1 & {P}_2{M}_2 & \hat{{L}}_z{M}_3& 0& 0&0&0&{I} \\
		* & * & -\gamma_1{I} & 0&0&0&0&0&0&0\\
		* & * &*  & -\gamma_2{I}&0&0&0&0&0&0\\
		*& * & * &*  & -\gamma_3{I}&0&0&0&0&0\\
		* &* & * & * & *&-\bar{\mu}_1{I}&0&0&0&0\\
		* & * & * &*& *&*&-\bar{\mu}_2{I}& 0&{N}_2\hat{{K}}_z&0\\
		*&*&* &* & * & * & *&-\bar{\mu}_3{I}&0&0\\
		* & * & *&*&* & *&*&*&-\frac{1}{\epsilon_1}{Q}_1&0\\
		* & * & * & *&*&*&*&*&*&-\epsilon_1{Q}_1\\
	\end{bmatrix}<0, \end{medsize}\]
%
где
%
\begin{align*}
	&\hat{{K}}_z={K}_z{Q}_1, \qquad \hat{{L}}_z={P}_2{L}_z, \\
	&{\Lambda}_1={Q}_1{A}_n\T+{A}_n{Q}_1+{B}\hat{{K}}_z+\hat{{K}}_z\T{B}\T, \\
	&{\Lambda}_2={A}_n\T{P}_2+{P}_2{A}_n-\hat{{L}}_z{C}{N}-{N}\T{C}\T\hat{{L}}_z\T, \\
	&{\Lambda}_3=\hat{{K}}_z\T{N}_2\T, \qquad {\Lambda}_4={Q}_1{N}\T{N}_3\T.
\end{align*}
	\end{block}
\end{frame}

\begin{frame}
	\frametitle{Управление с динамической обратной связью}
	Рассмотрим следующую систему:
	\begin{equation}
		\label{eq:part5_linear_dynamics}
		\begin{cases}
			\dot z={A}_n {z} + {A}_r {\zeta} + {B} {u},\\
			y = {C} {N} {z} + {C} {R} {\zeta},
		\end{cases}
	\end{equation}
	где матрицы $A_n$,  $B$ и $C$ являются матрицами состояния, управления и наблюдения соответственно. Матрица ${A}_r$ отвечает за статическую часть матрицы состояния.
	
	Рассмотрим следующий закон управления с динамической обратной связью:
	\begin{equation}
		\label{eq:part5_controller}
		\begin{cases}
			\dot{{z}}_K = {A}_K {z}_K + {B}_K {y},\\
			{u} = {C}_K {z}_K + {D}_K {y}.
		\end{cases}
	\end{equation}
\end{frame}

\begin{frame}
	\frametitle{Управление с динамической обратной связью}
		Объединим уравнения \eqref{eq:part5_linear_dynamics} и \eqref{eq:part5_controller} и получим систему:
	\begin{align}
		\label{eq:part5_cl}
		&\begin{bmatrix}
			{\dot{z}} \\ {\dot{z}}_K 
		\end{bmatrix}
		=
		\begin{bmatrix}
			{A}_N + {B}{D}_K{C}{N} & {B}{C}_K \\
			{B}_K{C}{N} &{A}_K
		\end{bmatrix}
		\begin{bmatrix}
			{z} \\ {z}_K 
		\end{bmatrix}
		+
		\begin{bmatrix}
			{A}_R + {B}{D}_K{C}{R}\\ {B}_K{C}_R
		\end{bmatrix}
		{\zeta}.
	\end{align}
	\begin{block}{Теорема 2}
		Система \eqref{eq:part5_cl}  асимптотически устойчива, если существуют положительно определённые матрицы ${Q}_1>0$, ${Q}_2>0$, ${P}_1>0$, матрицы $\hat{{A}}_K$,  $\hat{{B}}_K$, $\hat{{C}}_K$ и $D_K$
		и положительный скаляр $\alpha>0$ такие, что выполняется следующие линейные матричные неравенства:
		\[
		\begin{medsize}
			\begin{bmatrix}
				{\Theta}_1  & {A}_N + {\hat{A}}_K\T +{B}{D}_K{C}{N} & {A}_R + {B}{D}_K{C}{R}\\
				\cdots & {\Theta}_2 & {P}_{1}{A}_R + {P}_{1}{B}{D}_K{C}{R}+{P}_1{B}_K{C}{R}\\
				\cdots & \cdots & 0 
			\end{bmatrix} < 
			\begin{bmatrix}
				0 & 0\\
				0 & \alpha {I}
			\end{bmatrix},
		\end{medsize}\]
		\begin{align*}
			\begin{bmatrix} 
				{Q}_{1} & I \\ 
				I & {P}_{1}
			\end{bmatrix} > 0,
		\end{align*}
	\end{block}
\end{frame}

\begin{frame}
	\frametitle{Управление с динамической обратной связью}
	\begin{block}{Теорема 2. Продолжение}
	\begin{align*}
		&{\Theta}_1 = {A}_N{Q}_{1} +{Q}_{1}{A}_N\T + {B}{\hat{C}}_K +{\hat{C}}_K\T{B}\T, \\
		&{\Theta}_2 = {P}_{1}{A}_N +{A}_N\T{P}_{1}+{\hat{B}}_K{C}{N}+{N}\T{C}\T{\hat{B}}_K\T.
	\end{align*}
	
	И коэффициенты регулятора находятся следующим образом 
	\begin{align*}
		&{P}_{2} = ({I}-{P}_{1}{Q}_{1}){Q}_{2}^{-\top},\\
		&{C}_K = ({\hat{C}}_K - {D}_K {C} {N} {Q}_{1} ){Q}_{2}^{-\top},\\
		&{B}_K = {P}_{2}^{-1}({\hat{B}}_K- {P}_{1}{B}{D}_K),\\
		&{A}_K = {P}_{2}^{-1}({\hat{A}}_K-{P}_{1}({A}_N+{B}{D}_K{C}{N}){Q}_{1} -\\ 
		& - {P}_{2}{B}_K{C}{N}{Q}_{1}-{P}_{1} {B}{C}_K{Q}_{2}\T){Q}_{2}^{-\top}.
	\end{align*}
\end{block}
\end{frame}

\begin{frame}
	\frametitle{Описание шагающего робота}
	Используется плоская модель четвероногого робота \textit{Unitree A1} (робот описывается в одном горизонтальном и одном вертикальном измерении), как показано на рисунке \ref{fig:flat_robot},
	\begin{figure}
		\centering
		\includegraphics[width=0.7\linewidth]{images/FlatQuadruped.jpeg}
		\caption{Схема плоского четвероногого робота}
		\label{fig:flat_robot}
	\end{figure}
\end{frame}

\begin{frame}
	\frametitle{Целевая функция}
	Неопределённости ограничены следующим образом:
	%
	\begin{equation}
		{F}_i\T{F}_i\leq \nu_i {I}, \ \ i=1,2,3,
	\end{equation}
	где $\nu_i$ --- радиусы неопределённости --- скаляры, определяющие ограничения, накладываемые на нормы ${F}_i$. 
	
	Обозначим $\mu_i=\gamma_i\nu_i$ и $\bar{\mu}_i=\frac{1}{\mu_i}$ для $i=1,2,3$.
	Используя данные переменные, можно сформулировать следующие целевые функции:
	\begin{align}
		\label{eq:cost_lin}
		J_\text{lin} &= \sum_{i=1}^{3}\left(\bar{\mu}_i+\gamma_i\right), \\ 
		\label{eq:cost_quad}
		J_\text{quad} & =  \sum_{i=1}^{3}\left((\bar{\mu}_i-\varpi)^2+(\gamma_i-\varpi)^2\right) + \varpi^2,
	\end{align}
	%
	где $\varpi$ --- свободная переменная. Целью обеих целевых функций является максимизация $\nu_i$, которая достигается минимизацией либо $J_\text{lin}$, либо $J_\text{quad}$. Максимизируя $\nu_i$, достигается устойчивость к большему набору неопределённых матриц $\Delta {A}_n$, $\Delta {B}$ и $\Delta {C}$.
\end{frame}

\begin{frame}
	\frametitle{Применение методов}
	Выбираем линейную целевую функцию, и задача оптимизационного управления приобретает вид:
	%
	\begin{equation}
		\label{eq:thm4_OCP}
		\begin{aligned}
			& \underset{\bar{\mu}_i,\gamma_i,{Q}_1, {P}_2,\hat{{K}} , \hat{{L}} }{\text{минимизируя}}
			& &  \sum_{i=1}^{3}\left(\bar{\mu}_i+\gamma_i\right), \\
			& \text{при ограничениях}
			& & \begin{cases}
				{Q}_1>0, \ \
				{P}_2>0, \ \
				\bar{\mu}_i>0, \ \
				\gamma_i>0, \ \
				\text{для} \ \ i=1,2,3; \\
				\text{условие из Теоремы 1 }.
			\end{cases}
		\end{aligned}
	\end{equation}
	%
	Данная задача оптимизации c переменными $\bar{\mu}_i,\gamma_i,{Q}_1, {P}_2,\hat{{K}}, \hat{{L}}$ имеет параметр $\epsilon_1$, который мы можем найти решетчатым поиском.
\end{frame}

\begin{frame}
	\frametitle{Применение методов}
	\begin{figure}
		\centering
		\includegraphics[scale=0.85]{images/mult_soft_cost_qp_eps.JPG}
		\caption{Значение целевой функции задачи оптимального управления \eqref{eq:thm4_OCP}, решённой для плоского четвероногого робота, построенное относительно параметра $\epsilon_1$; обе оси без единиц.}
		\label{fig:cost}
	\end{figure}
\end{frame}

\begin{frame}
	\frametitle{Применение методов}
	\begin{figure}
		\centering
		\includegraphics[scale=0.8]{images/mult_soft_nu_eps_qp.JPG}
		\caption{Верхняя граница радиусов неопределённости $\nu_i$ задачи \eqref{eq:thm4_OCP}; графики построены относительно значения свободного параметра $\epsilon_1$; вертикальная ось масштабирована логарифмически, обе оси без единиц.}
		\label{fig:cost_qp}
	\end{figure}
\end{frame}

\begin{frame}
	\frametitle{Применение методов}
	Сформулируем оптимизационную задачу для системы с обратной динамической связью.
	Целевой функцией будет являться минимизация переменных матриц ${Q}_1, {Q}_2, {P}_1, {\hat{A}}_K, {\hat{B}}_K,{\hat{C}}_K, {D}_K$:
	\begin{align}
		\label{eq:cost_output}
		J_\text{out} &= \operatorname{tr}({Q}_1\T{W}_{q1}{Q}_1) + \operatorname{tr}({P}_1\T{W}_p{P}_1) + \operatorname{tr}({Q}_2\T{W}_{q2}{Q}_2) + \\
		 &\operatorname{tr}({\hat{A}}_K\T{W}_a{\hat{A}}_K) + \operatorname{tr}({\hat{B}}_K\T{W}_b{\hat{B}}_K) + \operatorname{tr}({\hat{C}}_K\T{W}_c{\hat{C}}_K) + \operatorname{tr}({D}_K\T{W}_d{D}_K). \nonumber
	\end{align}
	Cкаляр $\alpha$ также является переменной. Оптимизационная задача будет выглядеть следующим образом:
	%
	\begin{equation}
		\begin{aligned}
			\label{eq:thm2_OCP}
			& \underset{{Q}_1, {Q}_2, {P}_1, {\hat{A}}_K, {\hat{B}}_K,{\hat{C}}_K, {D}_K, \alpha }{\text{минимизируя}}
			& & J_\text{out}\\
			& \text{при ограничениях}
			& & \begin{cases}
				\text{условия из теоремы 2},
			\end{cases}
		\end{aligned}
	\end{equation}
	где ${W}_{q1}, {W}_p, {W}_{q2}, {W}_a, {W}_b, {W}_c$ и ${W}_d$ --- весовые матрицы.
\end{frame}

\begin{frame}
	\frametitle{Применение методов}
	\begin{figure}
	\centering
	\includegraphics[scale=0.85]{images/output_cost.JPG}
	\caption{Целевая функция для ЗОУ \eqref{eq:thm2_OCP}, решённые для плоского четвероногого робота, построенная относительно параметра $\alpha$; вертикальная ось масштабирована логарифмически, обе оси без единиц.}
	\label{fig:cost_output}
	\end{figure}
\end{frame}

\begin{frame}
	\frametitle{Применение методов}
	\begin{figure}
		\centering
		\includegraphics[width=0.8\linewidth]{images/output_state_z.JPG}
		\includegraphics[width=0.8\linewidth]{images/output_state_zk.JPG}
		\caption{Вектор состояния и вектор состояния регулятора от времени при решении \eqref{eq:thm2_OCP} для плоского четвероногого робота.}
		\label{fig:output_state_z}
	\end{figure}
\end{frame}

\subsection{Комплекс программ}

\begin{frame}
	\frametitle{Комплекс программ}
	\begin{figure}
	\centering
	\includegraphics[scale=0.2]{images/programm.png}
	\caption{Блок–схема программы с точки зрения пользователя.}
	\label{fig:programm}
\end{figure}
\end{frame}

\section{Заключение}




