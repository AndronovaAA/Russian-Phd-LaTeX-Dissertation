\begin{frame}
    \frametitle{Научная новизна}
\begin{enumerate}
	\item Впервые разработаны численные методы робастного оптимального управления с обратной связью, используя линейные матричные неравенства для систем с механическими ограничениями и параметрическими неопределённостями;
	\item Впервые разработаны численные методы робастного оптимального управления с динамической обратной связью, используя линейные матричные неравенства для систем с механическими ограничениями и аддитивными неопределённостями;
	\item Предложен метод математического моделирования для неопределённых систем, описывающих динамику шагающих роботов, включающий ортогональные методы и линеаризацию;
	\item Реализован комплекс программ в составе: а) моделирование мультипликативных неопределённостей в системе; б) реализация разработанных численных методов для вычисления оптимального робастного управления.
\end{enumerate}
\end{frame}

\begin{frame}
    \frametitle{Научная и практическая значимость}
	Теоретическая значимость данной работы состоит в разработке нового метода для вычисления коэффициентов регулятора и наблюдателя для робастного управления системы с мультипликативными неопределённостями с использованием линейных матричных неравенств. Разработанный численный метод для вычисления коэффициентов регулятора и наблюдателя вносит вклад в развитие теории выпуклой оптимизации и теории управления.
	
	Разработанные математические модели, численные методы и комплекс программ имеют практическую значимость для эксплуатации шагающих роботов. Практическая значимость результатов подтверждается применением результатов теоретической части в практике: приводятся формулировки задач оптимизационного управления для поиска коэффициентов регуляторов и наблюдателей для управления динамических систем, описывающих динамику шагающего робота. 
\end{frame}
\note{
    Проговариваются вслух научная и практическая значимость
}

\begin{frame} % публикации на одной странице
% \begin{frame}[t,allowframebreaks] % публикации на нескольких страницах
    \frametitle{Основные публикации}
    %% authorscopus
    \nocite{scbib1}%
 
    %% authorother
    \nocite{bib1}%
    \ifnumequal{\value{bibliosel}}{0}{
        \insertbiblioauthor
    }{
        \printbibliography%
    }
\end{frame}

\begin{frame}[plain, noframenumbering] % последний слайд без оформления
    \begin{center}
        \Huge
        Спасибо за внимание!
    \end{center}
\end{frame}
