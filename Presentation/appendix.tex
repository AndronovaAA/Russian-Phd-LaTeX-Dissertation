\begin{frame}
    \frametitle{Математические инструменты}
\emph{S--процедура:}
Рассмотрим ${M},{N} \in \mathbb{R}^{n\times n}$. Если существует скаляр $\gamma>0$, который отвечает следующему условию ${M}+\gamma {N}<0$, тогда $x\T {N} x\geq 0$ подразумевает $x\T{M}x\leq 0$.

\emph{Неравенство Юнга:}
Для любой положительно определённой матрицы $M>0$, положительных скаляров $\epsilon > 0, \nu > 0$ и матриц ${X}, {Y}, {F}$, где ${F}\T{F}\leq \nu{I}$, следующие неравенства верны:
%
\begin{align}
	\label{eq:Young_relation_BMI}
	{X}\T{Y} + {Y}\T{X}  \leq {X}\T 
	\epsilon {M}^{-1}{X} + \frac{1}{\epsilon}   {Y}\T  {M}{Y}, 
	\\
	\label{eq:Young_robust_2}
	{X}\T{F}{Y} + {Y}\T{F}\T{X}  \leq \epsilon {X}\T{X} +  \frac{\nu}{\epsilon} {Y}\T {Y}.
\end{align}
%
Обозначим $\nu=\epsilon^2$ для формулирования следующего неравенства:
%
\begin{equation}
	\label{eq:updated_Young_robust_2}
	{X}^T{F}{Y} + {Y}\T{F}\T{X}  \leq \epsilon {X}\T{X} + \epsilon {Y}\T{Y}.
\end{equation}
\end{frame}

\begin{frame}
	\frametitle{Математические инструменты}
	\emph{Дополнение Шура:}
	Для любой симметричной матрицы ${A}\in \mathbb{S}^n$ и ${C}\in \mathbb{S}^m$ и матрицы ${B}\in \mathbb{R}^{n\times m}$, а также их конкатенации:
	\noindent \begin{align*}
		{M}= \begin{bmatrix}
			{A} & {B} \\
			{B}\T & {C} 
		\end{bmatrix},
	\end{align*}
	%
	следующие утверждения равнозначны:
	% 
	\noindent
	\begin{enumerate}
		\item ${M} < 0$,
		\item ${A}-{B}{C}^{-1}{B}\T < 0 , {C}< 0$,
		\item ${C}-{B}\T{A}^{-1}{B}< 0 , {A}< 0$.
	\end{enumerate}
\end{frame}

\begin{frame}
	\frametitle{Системы с механическими ограничениями}
    \begin{figure}
	\begin{minipage}[c]{0.4\textwidth}
		\textbf{Ортогональные методы}
		\begin{equation*}
			\begin{gathered}
				\dot{x}=Ax+Bu+F\lambda\\
				G\dot{x}=0
			\end{gathered}
		\end{equation*}
	\end{minipage}\hfill
	\begin{minipage}[c]{0.6\textwidth}
		\textbf{Дескрипторные методы}
		\begin{equation*}
			\begin{gathered}
				E\dot{x}=Ax+Bu\\
				\begin{pmatrix}
					I & 0\\
					G & 0
				\end{pmatrix}
				\begin{pmatrix}
					\dot{x}\\ \dot{\lambda}
				\end{pmatrix}=
				\begin{pmatrix}
					A & F\\ 0 & 0
				\end{pmatrix}
				\begin{pmatrix}
					x\\ \lambda
				\end{pmatrix}+
				\begin{pmatrix}
					B \\ 0
				\end{pmatrix}
				u
			\end{gathered}
		\end{equation*}
	\end{minipage}
\end{figure}
\end{frame}
