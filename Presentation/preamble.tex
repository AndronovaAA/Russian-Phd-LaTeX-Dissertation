\begin{frame}[noframenumbering,plain]
    \setcounter{framenumber}{1}
    \maketitle
\end{frame}

\begin{frame}
    \frametitle{Положения, выносимые на защиту}
    \begin{itemize}
        \item Численный метод оптимального робастного управления для системы с параметрическими неопределённостями. Позволяет находить оптимальные коэффициенты регулятора и наблюдателя для различных неопределённостей в системе, удовлетворяющих заданным условиям и с использованием линейных матричных неравенств.
        \item Численный метод оптимального робастного управления для системы с параметрической неопределённостью и с динамической обратной связью. Позволяет находить оптимальные коэффициенты регулятора для различных неопределённостей, удовлетворяющих заданным условиям и с использованием линейных матричных неравенств.
        \item Комплекс программных средств в составе: а) моделирование неопределённостей в системе; б) реализация разработанных численных методов.
    \end{itemize}
\end{frame}
\note{
    Проговариваются вслух положения, выносимые на защиту
}

\begin{frame}
    \frametitle{Содержание}
    \tableofcontents
\end{frame}
\note{
    Работа состоит из четырёх глав.

    \medskip
    В первой главе \dots

    Во второй главе \dots

    Третья глава посвящена \dots

    В четвёртой главе \dots
}
